\documentclass[10pt]{article}

\usepackage[english]{babel}
\usepackage[utf8x]{inputenc}
\usepackage{amsmath}
\usepackage{amssymb}
\usepackage{amsfonts}
\usepackage{graphicx}
\usepackage[ruled,linesnumbered,noend]{algorithm2e}
\usepackage{empheq}
\usepackage{float}
\usepackage{enumitem}
\usepackage{tikz}
\usepackage[colorlinks=true,urlcolor=blue]{hyperref}
\usepackage{braket}
\usepackage{gensymb}

\title{Kvanttilaskenta, kevät 2015 -- Viikko 2}
\author{Rodion ``rodde'' Efremov}

\begin{document}
 \maketitle

\section*{edx Problem 1}
No, its 0 for $\ket{+}$ and $\ket{-}$ are orthogonal to each other.

\section*{edx Problem 2}
Yes.

\section*{edx Problem 3}
Yes, but the implicit multiplication operator is in fact the tensor product $\otimes$.
 
\section*{edx Problem 4}
False.

\section*{edx Problem 5}
False.

\section*{edx Problem 6}
True.

\section*{edx Problem 7}
$\ket{+} = \frac{1}{\sqrt{2}} \ket{0} + \frac{1}{\sqrt{2}} \ket{1}$. $\ket{\psi} = \frac{3}{5} \ket{0} - \frac{4}{5} \ket{1}$. Now we wish to compute
\begin{align*}
\braket{\psi, +} &= \frac{3}{5} \frac{1}{\sqrt{2}} - \frac{4}{5} \frac{1}{\sqrt{2}} \\
 						  &= -\frac{1}{\sqrt{2}} \frac{1}{5} \\
 						  &= -\frac{1}{5\sqrt{2}}.
\end{align*}

\section*{edx Problem 8}
\[
-\frac{1}{5 \sqrt{2} } \ket{+} + \frac{7}{5 \sqrt{2}} \ket{-}.
\]

\section*{edx Problem 9}
Forget the first standard basis measurement as it is not relevant here. At second measurement the chance of getting $\ket{u}$ is $\cos^2 \theta$, where $\theta$ is the angle between $\ket{u}$ and $\ket{\phi}$. Since
\[
\cos \theta = \braket{\phi | u} = ab + \sqrt{1 - a^2} \sqrt{1 - b^2},
\]
the desired probability is 
\[
(ab + \sqrt{1 - a^2} \sqrt{1 - b^2})^2.
\]

\section*{edx Problem 10}
The third topmost alternative seems suspicious as it mixes sign and bit bases.

\section*{edx Problem 11}
\begin{align*}
\Bigg( \frac{\sqrt{2}}{\sqrt{3}} \ket{0} + \frac{1}{\sqrt{3}} \ket{1} \Bigg) \otimes \Bigg( \frac{1}{\sqrt{3}} \ket{0} + \frac{\sqrt{2}}{\sqrt{3}} \ket{1} \Bigg) &= \frac{\sqrt{2}}{3} \ket{00} + \frac{2}{3} \ket{01} + \frac{1}{3} \ket{10} + \frac{\sqrt{2}}{3} \ket{11} 
\end{align*}

\section*{edx Problem 12}
\begin{align*}
(a \ket{0} + b \ket{1})(c \ket{0} + d \ket{1}) &= ac \ket{00} + ad \ket{01} + bc \ket{10} + bd \ket{11} \\
																	 &= \frac{1}{2\sqrt{2}} \ket{00} - \frac{1}{2\sqrt{2}} \ket{01} + \frac{\sqrt{3}}{2\sqrt{2}} \ket{10} - \frac{\sqrt{3}}{2\sqrt{2}} \ket{11} 
\end{align*}
So we have that 
\begin{align*}
ac &= \frac{1}{2\sqrt{2}}, \\
ad &= -\frac{1}{2\sqrt{2}}, \\
bc &= \frac{\sqrt{3}}{2\sqrt{2}}, \\
bd &= -\frac{\sqrt{3}}{2\sqrt{2}}.
\end{align*}
The way to factorize is to assign $a = \frac{1}{2}$, $b = \frac{\sqrt{3}}{2}$, $c = \frac{1}{\sqrt{2}}$, $d = -\frac{1}{\sqrt{2}}$, so $|a| = \frac{1}{2}$.

\section*{edx Problem 13}
\subsection*{(a)} 
The probability is $\frac{1}{2}$.

\subsection*{(b)}
$\ket{+} = \frac{1}{\sqrt{2}} \ket{0} + \frac{1}{\sqrt{2}} \ket{1}$.

\section*{edx Problem 14}
\[
\Bigg( \frac{4}{5} \Bigg)^2 + \Bigg( -\frac{2}{5} \Bigg)^2 = \frac{16}{25} + \frac{4}{25} = \frac{20}{25}.
\]

\section*{edx Problem 15}
$\ket{0+}, \ket{0-}, \ket{1+}, \ket{1-}$.

\section*{edx Problem 16}
There si no way to entangle two qubits by a partial measurement.

\section*{edx Problem 17}
As 
\[
\ket{0} = \frac{1}{\sqrt{2}} \ket{+} + \frac{1}{\sqrt{2}} \ket{-}, \ket{1} = \frac{1}{\sqrt{2}} \ket{+} - \frac{1}{\sqrt{2}} \ket{-},
\]
we have that
\begin{align*}
\ket{\psi} &= \frac{1}{\sqrt{2}} \ket{0} + \frac{e^{i\theta}}{\sqrt{2}} \ket{1} \\
                &= \frac{1}{\sqrt{2}} \Bigg( \frac{1}{\sqrt{2}} \ket{+} + \frac{1}{\sqrt{2}} \ket{-} \Bigg) + \frac{e^{i\theta}}{\sqrt{2}} \Bigg( \frac{1}{\sqrt{2}} \ket{+} - \frac{1}{\sqrt{2}} \ket{-} \Bigg) \\
                &= \frac{1}{2} \ket{+} + \frac{1}{2} \ket{-} + \frac{e^{i \theta}}{2} \ket{+} - \frac{e^{i \theta}}{2} \ket{-} \\
                &= \frac{1 + e^{i \theta}}{2} \ket{+} + \frac{1 - e^{i \theta}}{2} \ket{-}.
\end{align*}

\section*{edx Problem 18}
\[
\frac{1 + \cos \theta}{2}.
\]

\section*{QCE 3.1}
$\ket{\psi} = \alpha \ket{0} + \beta \ket{1}$. $X = \ket{0} \bra{1} + \ket{1} \bra{0}$. $Y  = -i\ket{0} \bra{1} + i\ket{1} \bra{0}$. Now
\begin{align*}
X\ket{\psi} &= (\ket{0} \bra{1} + \ket{1} \bra{0}) (\alpha \ket{0} + \beta \ket{1}) \\
                  &= \alpha ( \ket{0} \bra{1} + \ket{1} \bra{0}) \ket{0} + \beta ( \ket{0} \bra{1} + \ket{1} \bra{0} ) \ket{1} \\
                  &= \alpha (\ket{0} \braket{ 1 | 0 } + \ket{1} \braket{ 0 | 0 }) + \beta ( \ket{0} \braket{ 1 | 1 } + \ket{1}  \braket{ 0 | 1 }) \\
                  &= \alpha \ket{0} \times 0 + \alpha \ket{1} \times 1 + \beta \ket{0} \times 1 + \beta \ket{1} \times 0 \\
                  &= \alpha \ket{1} + \beta \ket{0}.
\end{align*}
\begin{align*}
Y \ket{\psi} &= (-i \ket{0} \bra{1} + i \ket{1} \bra{0}) (\alpha \ket{0} + \beta \ket{1}) \\
                  &= \alpha (-i \ket{0} \bra{1} + i \ket{1} \bra{0}) \ket{0} + \beta ( -i \ket{0} \bra{1} + i \ket{1} \bra{0} ) \ket{1} \\
                  &= \alpha ( -i \ket{0} \braket{ 1 | 0 } + i \ket{1} \braket{ 0 | 0 } ) + \beta ( -i \ket{0} \braket{ 1 | 1 } + i \ket{1} \braket{ 0 | 1 }  ) \\
                  &= \alpha i \ket{1} + \beta (-i) \ket{0} \\
                  &= \alpha i \ket{1} - \beta i \ket{0}.
\end{align*}

\section*{QCE 3.2}
Suppose we have a qubit 
\[
\begin{pmatrix}
\alpha \\
\beta
\end{pmatrix}.
\]
Now 
\begin{align*}
\begin{pmatrix}
0 & 1 \\
1 & 0 
\end{pmatrix}
\begin{pmatrix}
\alpha \\
\beta
\end{pmatrix} = 
\begin{pmatrix}
0 \times \alpha + 1 \times \beta \\
1 \times \alpha + 0 \times \beta
\end{pmatrix} = 
\begin{pmatrix}
\beta \\
\alpha
\end{pmatrix}.
\end{align*}

\section*{QCE 3.3}
\begin{align*}
\ket{+} = \frac{\ket{0} + \ket{1}}{\sqrt{2}}, \ket{-} = \frac{\ket{0} - \ket{1}}{\sqrt{2}}.
\end{align*}
Now
\begin{align*}
X \ket{+} &= X \frac{\ket{0} + \ket{1}}{\sqrt{2}} \\
               &= X \begin{pmatrix}
               \frac{1}{\sqrt{2}} \\
               \frac{1}{\sqrt{2}}
               \end{pmatrix} \\
               &= \begin{pmatrix}
               1 & 0 \\
               0 & - 1 
               \end{pmatrix} \begin{pmatrix}
               \frac{1}{\sqrt{2}} \\
               \frac{1}{\sqrt{2}}
               \end{pmatrix} \\
               & = \begin{pmatrix}
               1 \times \frac{1}{\sqrt{2}} + 0 \times \frac{1}{\sqrt{2}} \\
               0 \times \frac{1}{\sqrt{2}} - 1 \frac{1}{\sqrt{2}}
               \end{pmatrix} \\
                &= \begin{pmatrix}
               \frac{1}{\sqrt{2}} \\
               - \frac{1}{\sqrt{2}}
               \end{pmatrix} \\
               &= \frac{\ket{0} - \ket{1}}{\sqrt{2}} \\
               &= \ket{-}.
\end{align*}
Also 
\begin{align*}
X \ket{-} &= X \frac{\ket{0} - \ket{1}}{\sqrt{2}} \\
               &= X \begin{pmatrix}
               \frac{1}{\sqrt{2}} \\
               - \frac{1}{\sqrt{2}}
               \end{pmatrix} \\
               &= \begin{pmatrix}
               1 & 0 \\
               0 & - 1 
               \end{pmatrix} \begin{pmatrix}
               \frac{1}{\sqrt{2}} \\
               - \frac{1}{\sqrt{2}}
               \end{pmatrix} \\
               & = \begin{pmatrix}
               1 \times \frac{1}{\sqrt{2}} + 0 \times -\frac{1}{\sqrt{2}} \\
               0 \times \frac{1}{\sqrt{2}} - 1 \Bigg(-\frac{1}{\sqrt{2}} \Bigg)
               \end{pmatrix} \\
                &= \begin{pmatrix}
               \frac{1}{\sqrt{2}} \\
               \frac{1}{\sqrt{2}}
               \end{pmatrix} \\
               &= \frac{\ket{0} + \ket{1}}{\sqrt{2}} \\
               &= \ket{+}.
\end{align*}

\section*{QCE 3.4} 
The operator is $\hat{A} = i \ket{1} \bra{1} + \frac{\sqrt{3}}{2} \ket{1} \bra{2} + 2 \ket{2} \ket{1} - \ket{2} \bra{3}$. Now, 
\begin{align*}
\hat{A}^{\dag} &= -i \braket{1 | 1} + \frac{\sqrt{3}}{2} \braket{ 1 | 2 } + 2 \braket{2 | 1} - \braket{2 | 3} \\
                       &= -i.
\end{align*}

\section*{QCE 3.5}
The X operator is given by the matrix
\begin{align*}
\begin{pmatrix}
0 & 1 \\
1 & 0
\end{pmatrix}
\end{align*}
and we wish to find $\lambda$, $(a \quad b)^T$ such that
\begin{align*}
\begin{pmatrix}
0 & 1 \\
1 & 0
\end{pmatrix}
\begin{pmatrix}
a \\
b
\end{pmatrix} =
\lambda \begin{pmatrix}
a \\
b
\end{pmatrix}.
\end{align*}
Now
\begin{align*}
\begin{pmatrix}
b \\
a
\end{pmatrix} = \lambda \begin{pmatrix}
a \\
b
\end{pmatrix},
\end{align*}
so $b = \lambda a$ and $a = \lambda b$. If $\lambda = 1$, the corresponding eigenvector is $(1, 1)^T$. If $\lambda = 0$, the eigenvector is $(0, 0)^T$.

\section*{QCE 3.6}
As the matrix of Y-operator is
\[
\begin{pmatrix}
0 & -i \\
i & 0
\end{pmatrix},
\]
it is evident that its trace is 0.

\section*{QCE 3.7}
\begin{align*}
\begin{pmatrix}
1 & 0 & 2 \\
0 & 3 & 4 \\
1 & 0 & 2
\end{pmatrix} \begin{pmatrix}
a \\
b \\
c \\
\end{pmatrix} = \begin{pmatrix}
a + 2c \\
3b + 4c \\
a + 2c
\end{pmatrix} = 
\begin{pmatrix}
\lambda a \\
\lambda b \\
\lambda c
\end{pmatrix}.
\end{align*}
Its clear that $a = b$. Also $3b + 4c = \lambda b$, which implies $4c = (\lambda - 3)b$, and $c = \frac{1}{4}(\lambda - 3)b$. It follows that $\lambda \in \mathbb{R}$.

\section*{QCE 3.8}
Suppose
\[
A = \begin{pmatrix}
a_{1, 1} & \dots & a_{1,n} \\
\vdots & \ddots & \vdots \\
a_{n, 1} & \dots & a_{n, n}
\end{pmatrix},
B = \begin{pmatrix}
b_{1, 1} & \dots & b_{1,n} \\
\vdots & \ddots & \vdots \\
b_{n, 1} & \dots & b_{n, n}
\end{pmatrix}.
\]
 Now, it is easy to see that 
\begin{align*}
Tr(A + B) &= (a_{1, 1} + b_{1, 1}) + \dots + (a_{n,n} + b_{n, n}) \\
               &= (a_{1,1} + \dots + a_{n,n})  + (b_{1,1} + \dots + b_{n,n}) \\
               &= Tr(A) + Tr(B).
\end{align*}
Also
\begin{align*}
  Tr(\lambda A) &= \lambda a_{1,1} + \dots + \lambda a_{n,n} \\
                        &= \lambda Tr(A).
\end{align*}
Also-also
\begin{align*}
Tr(AB) &= a_{1, 1}b_{1,1} + \dots + a_{n,n}b_{n,n} \\
          &= b_{1, 1}a_{1,1} + \dots + b_{n,n}a_{n,n} \\
          &= Tr(BA),
\end{align*}
as real multiplication commutes.

\section*{QCE 3.9}
\[
P_+ = \frac{1}{2} ( \ket{0} \bra{0} + \ket{0} \bra{1} +\ket{1} \bra{0} +\ket{1} \bra{1}).
\]
\[
P_- = \frac{1}{2} (\ket{0} \bra{0} - \ket{0} \bra{1} -\ket{1} \bra{0} +\ket{1} \bra{1}).
\]
Now 
\[
P_+ - P_- = \ket{0} \bra{1} + \ket{1} \bra{0} = X.
\]

\section*{QCE 3.10}
We have
\[
P_0 = \ket{0} \bra{0} = \begin{pmatrix}
1 \\
0 \\
0
\end{pmatrix} \big( 1 \; 0 \; 0 \big) = \begin{pmatrix}
1 & 0 & 0 \\
0 & 0 & 0 \\
0 & 0 & 0 
\end{pmatrix},
\]
\[
P_1 = \ket{1} \bra{1} = \begin{pmatrix}
0 \\
1 \\
0
\end{pmatrix} \big( 0 \; 1 \; 0 \big) = \begin{pmatrix}
0 & 0 & 0 \\
0 & 1 & 0 \\
0 & 0 & 0 
\end{pmatrix},
\]
\[
P_2 = \ket{2} \bra{2} = \begin{pmatrix}
0 \\
0 \\
1
\end{pmatrix} \big( 0 \; 0 \; 1 \big) = \begin{pmatrix}
0 & 0 & 0 \\
0 & 0 & 0 \\
0 & 0 & 1 
\end{pmatrix}.
\]
Now 
\[
Pr(0) = |P_0 \ket{\psi}|^2 = \Bigg| \begin{pmatrix}
1 & 0 & 0 \\
0 & 0 & 0 \\
0& 0 & 0 
\end{pmatrix} \begin{pmatrix}
\frac{1}{2} \\
\frac{1}{2} \\
-\frac{i}{\sqrt{2}}
\end{pmatrix} \Bigg|^2 = \Bigg| \begin{pmatrix}
\frac{1}{2} \\
0 \\
0
\end{pmatrix} \Bigg|^2 = \Bigg( \frac{1}{2} \Bigg)^2 = \frac{1}{4},
\]
\[
Pr(1) = |P_1 \ket{\psi}|^2 = \Bigg| \begin{pmatrix}
0 & 0 & 0 \\
0 & 1 & 0 \\
0& 0 & 0 
\end{pmatrix} \begin{pmatrix}
\frac{1}{2} \\
\frac{1}{2} \\
-\frac{i}{\sqrt{2}}
\end{pmatrix} \Bigg|^2 = \Bigg| \begin{pmatrix}
0 \\
\frac{1}{2} \\
0 
\end{pmatrix} \Bigg|^2 = \Bigg( \frac{1}{2} \Bigg)^2 = \frac{1}{4},
\]
\[
Pr(2) = |P_2 \ket{\psi}|^2 = \Bigg| \begin{pmatrix}
0 & 0 & 0 \\
0 & 0 & 0 \\
0& 0 & 1 
\end{pmatrix} \begin{pmatrix}
\frac{1}{2} \\
\frac{1}{2} \\
-\frac{i}{\sqrt{2}}
\end{pmatrix} \Bigg|^2 = \Bigg| \begin{pmatrix}
0 \\
0 \\
-\frac{i}{\sqrt{2}}
\end{pmatrix} \Bigg|^2 = \Bigg( \frac{1}{\sqrt{2}} \Bigg)^2 = \frac{1}{2}.
\]

\section*{QCE 3.11}
The matrix representations of the Pauli operators are 
\[
\sigma_1 = \begin{pmatrix}
0 & 1 \\
1 & 0
\end{pmatrix},
\sigma_2 = \begin{pmatrix}
0 & -i \\
i & 0
\end{pmatrix}, 
\sigma_3 = \begin{pmatrix}
1 & 0 \\
0 & -1.
\end{pmatrix}
\]
\subsection{$[\sigma_2, \sigma_3] = 2i\sigma_1$}
\[
\sigma_2 \sigma_3 = \begin{pmatrix}
0 &  -i \\
i & 0
\end{pmatrix} 
\begin{pmatrix}
1 & 0 \\
0 & -1
\end{pmatrix} =
\begin{pmatrix}
0 & i \\
i & 0
\end{pmatrix},
\]
%
\[
\sigma_3 \sigma_2 = 
\begin{pmatrix}
1 & 0 \\
0 & -1
\end{pmatrix} 
\begin{pmatrix}
0 &  -i \\
i & 0
\end{pmatrix} 
=
\begin{pmatrix}
0 & -i \\
-i & 0 
\end{pmatrix}.
\]
Now the commutator is 
\[
[\sigma_2, \sigma_3] = \sigma_2\sigma_3 - \sigma_3\sigma_2 =
\begin{pmatrix}
0 & i \\
i & 0
\end{pmatrix}
-
\begin{pmatrix}
0 & -i \\
-i & 0
\end{pmatrix}
=
\begin{pmatrix}
0 & 2i \\
2i & 0
\end{pmatrix} = 2i\sigma_1.
\]

\subsection{$[\sigma_3, \sigma_1] = 2i\sigma_2$}
\[
\sigma_3 \sigma_1 = 
\begin{pmatrix}
1 & 0 \\
0 & -1
\end{pmatrix}
\begin{pmatrix}
0 & 1 \\
1 & 0 
\end{pmatrix}
=
\begin{pmatrix}
0 & 1 \\
-1 & 0
\end{pmatrix},
\]
%
\[ 
\sigma_1 \sigma_3 = 
\begin{pmatrix}
0 & 1 \\
1 & 0 
\end{pmatrix}
\begin{pmatrix}
1 & 0 \\
0 & -1
\end{pmatrix}
=
\begin{pmatrix}
0 & -1 \\
1 & 0
\end{pmatrix}.
\]
Now the commutator is 
\[
[\sigma_3, \sigma_1] = \sigma_3 \sigma_1 - \sigma_1 \sigma_3 =
\begin{pmatrix}
0 & 1 \\
-1 & 0 
\end{pmatrix}
-
\begin{pmatrix}
0 & - 1 \\
1 & 0 
\end{pmatrix}
= 
\begin{pmatrix}
0 & 2 \\
-2 & 0
\end{pmatrix}
= 2i\sigma2.
\]

\section*{QCE 3.12}
We need to show that if $i \neq j$, then $\{ \sigma_i, \sigma_j \} = 0$. First of all, $\{ \sigma_i, \sigma_j \} = \sigma_i \sigma_j + \sigma_j \sigma_i$. By the definition of anticommutator we have that
\begin{align*}
\{ \sigma_i, \sigma_j \} = \sigma_i \sigma_j + \sigma_j \sigma_i = \sigma_j \sigma_i + \sigma_i \sigma_j = \{ \sigma_j , \sigma_i \},
\end{align*} 
because the real $+$ commutes. Now we don't need to compute all permutations of Pauli operators, but only combinations will suffice, so we need to calculate
\begin{itemize}
\item $\{ \sigma_1, \sigma_2 \}$,
\item $\{ \sigma_1, \sigma_3 \}$,
\item $\{ \sigma_2, \sigma_3 \}$.
\end{itemize}
So
\begin{align*}
\{ \sigma_1, \sigma_2 \} &= 
\begin{pmatrix}
0 & 1 \\
1 & 0
\end{pmatrix}
\begin{pmatrix}
0 & -i \\
i & 0
\end{pmatrix} + \begin{pmatrix}
0 & -i \\
i & 0
\end{pmatrix}
\begin{pmatrix}
0 & 1 \\
1 & 0
\end{pmatrix} \\
   &= 
   \begin{pmatrix}
   i & 0 \\
   0 & -i
   \end{pmatrix} +
   \begin{pmatrix}
   -i & 0 \\
   0 & i
   \end{pmatrix} \\
   &= 0.
\end{align*}
\begin{align*}
\{ \sigma_1, \sigma_3 \} &= 
\begin{pmatrix}
0 & 1 \\
1 & 0
\end{pmatrix}
\begin{pmatrix}
1 & 0 \\
0 & -1
\end{pmatrix} + \begin{pmatrix}
1 & 0 \\
0 & -1
\end{pmatrix}
\begin{pmatrix}
0 & 1 \\
1 & 0
\end{pmatrix} \\
   &= 
   \begin{pmatrix}
   0 & -1 \\
   1 & 0
   \end{pmatrix} +
   \begin{pmatrix}
   0 & 1 \\
   -1 & 0
   \end{pmatrix} \\
   &= 0.
\end{align*}
The last, but not least
\begin{align*}
\{ \sigma_2, \sigma_3 \} &= 
\begin{pmatrix}
0 & -i \\
i & 0
\end{pmatrix}
\begin{pmatrix}
1 & 0 \\
0 & -1
\end{pmatrix} + \begin{pmatrix}
1 & 0 \\
0 & -1
\end{pmatrix}
\begin{pmatrix}
0 & -i \\
i & 0
\end{pmatrix} \\
   &= 
   \begin{pmatrix}
   0 & i \\
   i & 0
   \end{pmatrix} +
   \begin{pmatrix}
   0 & -i \\
   -i & 0
   \end{pmatrix} \\
   &= 0.
\end{align*}

\end{document}