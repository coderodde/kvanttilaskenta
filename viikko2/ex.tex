\documentclass[10pt]{article}

\usepackage[english]{babel}
\usepackage[utf8x]{inputenc}
\usepackage{amsmath}
\usepackage{amssymb}
\usepackage{amsfonts}
\usepackage{graphicx}
\usepackage[ruled,linesnumbered,noend]{algorithm2e}
\usepackage{empheq}
\usepackage{float}
\usepackage{enumitem}
\usepackage{tikz}
\usepackage[colorlinks=true,urlcolor=blue]{hyperref}
\usepackage{braket}
\usepackage{gensymb}

\title{Kvanttilaskenta, kevät 2015 -- Viikko 2}
\author{Rodion ``rodde'' Efremov}

\begin{document}
 \maketitle

\section*{edx Problem 1}
No, its 0 for $\ket{+}$ and $\ket{-}$ are orthogonal to each other.

\section*{edx Problem 2}
Yes.

\section*{edx Problem 3}
Yes, but the implicit multiplication operator is in fact the tensor product $\otimes$.
 
\section*{edx Problem 4}
False.

\section*{edx Problem 5}
False.

\section*{edx Problem 6}
True.

\section*{edx Problem 7}
$\ket{+} = \frac{1}{\sqrt{2}} \ket{0} + \frac{1}{\sqrt{2}} \ket{1}$. $\ket{\psi} = \frac{3}{5} \ket{0} - \frac{4}{5} \ket{1}$. Now we wish to compute
\begin{align*}
\braket{\psi, +} &= \frac{3}{5} \frac{1}{\sqrt{2}} - \frac{4}{5} \frac{1}{\sqrt{2}} \\
 						  &= -\frac{1}{\sqrt{2}} \frac{1}{5} \\
 						  &= -\frac{1}{5\sqrt{2}}.
\end{align*}

\section*{edx Problem 8}
\[
-\frac{1}{5 \sqrt{2} } \ket{+} + \frac{7}{5 \sqrt{2}} \ket{-}.
\]

\section*{edx Problem 9}
Forget the first standard basis measurement as it is not relevant here. At second measurement the chance of getting $\ket{u}$ is $\cos^2 \theta$, where $\theta$ is the angle between $\ket{u}$ and $\ket{\phi}$. Since
\[
\cos \theta = \braket{\phi | u} = ab + \sqrt{1 - a^2} \sqrt{1 - b^2},
\]
the desired probability is 
\[
(ab + \sqrt{1 - a^2} \sqrt{1 - b^2})^2.
\]

\section*{edx Problem 10}
The third topmost alternative seems suspicious as it mixes sign and bit bases.

\section*{edx Problem 11}
\begin{align*}
\Bigg( \frac{\sqrt{2}}{\sqrt{3}} \ket{0} + \frac{1}{\sqrt{3}} \ket{1} \Bigg) \otimes \Bigg( \frac{1}{\sqrt{3}} \ket{0} + \frac{\sqrt{2}}{\sqrt{3}} \ket{1} \Bigg) &= \frac{\sqrt{2}}{3} \ket{00} + \frac{2}{3} \ket{01} + \frac{1}{3} \ket{10} + \frac{\sqrt{2}}{3} \ket{11} 
\end{align*}

\section*{edx Problem 12}
\begin{align*}
(a \ket{0} + b \ket{1})(c \ket{0} + d \ket{1}) &= ac \ket{00} + ad \ket{01} + bc \ket{10} + bd \ket{11} \\
																	 &= \frac{1}{2\sqrt{2}} \ket{00} - \frac{1}{2\sqrt{2}} \ket{01} + \frac{\sqrt{3}}{2\sqrt{2}} \ket{10} - \frac{\sqrt{3}}{2\sqrt{2}} \ket{11} 
\end{align*}
So we have that 
\begin{align*}
ac &= \frac{1}{2\sqrt{2}}, \\
ad &= -\frac{1}{2\sqrt{2}}, \\
bc &= \frac{\sqrt{3}}{2\sqrt{2}}, \\
bd &= -\frac{\sqrt{3}}{2\sqrt{2}}.
\end{align*}
The way to factorize is to assign $a = \frac{1}{2}$, $b = \frac{\sqrt{3}}{2}$, $c = \frac{1}{\sqrt{2}}$, $d = -\frac{1}{\sqrt{2}}$, so $|a| = \frac{1}{2}$.
\end{document}