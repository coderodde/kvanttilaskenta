\documentclass[10pt]{article}

\usepackage[english]{babel}
\usepackage[utf8x]{inputenc}
\usepackage{amsmath}
\usepackage{amssymb}
\usepackage{amsfonts}
\usepackage{graphicx}
\usepackage[ruled,linesnumbered,noend]{algorithm2e}
\usepackage{empheq}
\usepackage{float}
\usepackage{enumitem}
\usepackage{tikz}
\usepackage[colorlinks=true,urlcolor=blue]{hyperref}
\usepackage{braket}
\usepackage{gensymb}

\title{Kvanttilaskenta, kevät 2015 -- Viikko 4}
\author{Rodion ``rodde'' Efremov}

\begin{document}
 \maketitle

\section*{edx Problem 1}
False. By definition of reversibility we should have $x$ at the output of $R_f$.

\section*{edx Problem 2}
False. The quantum circuit should modify each $\alpha_x$.

\section*{edx Problem 3}
True. Straight from the slides.

\section*{edx Problem 4}
\begin{align*}
H^{\otimes 2} \frac{1}{\sqrt{2}} \ket{00} &= \frac{1}{\sqrt{2}} \ket{++} \\
                                        						&= \frac{1}{\sqrt{2}} \Bigg( \frac{1}{\sqrt{2}} \ket{0} + \frac{1}{\sqrt{2}} \ket{1} \Bigg)\Bigg( \frac{1}{\sqrt{2}} \ket{0} + \frac{1}{\sqrt{2}} \ket{1} \Bigg) \\
                                        						&= \frac{1}{2\sqrt{2}} \Bigg( \ket{00} + \ket{01} + \ket{10} + \ket{11} \Bigg).
\end{align*}
Also
\begin{align*}
H^{\otimes 2} \frac{1}{\sqrt{2}} \ket{11} &= \frac{1}{\sqrt{2}} \ket{--} \\
                                        						&= \frac{1}{\sqrt{2}} \Bigg( \frac{1}{\sqrt{2}} \ket{0} - \frac{1}{\sqrt{2}} \ket{1} \Bigg)\Bigg( \frac{1}{\sqrt{2}} \ket{0} - \frac{1}{\sqrt{2}} \ket{1} \Bigg) \\
                                        						&= \frac{1}{2\sqrt{2}} \Bigg( \ket{00} - \ket{01} - \ket{10} + \ket{11} \Bigg),
\end{align*}
so
\begin{align*}
H^{\otimes 2}\psi &= H^{\otimes 2}\Bigg( \frac{1}{\sqrt{2}} \ket{00} + \frac{1}{\sqrt{2}} \ket{11} \Bigg) \\
 							  &= \frac{1}{2\sqrt{2}} \Bigg( 2\ket{00}  + 2\ket{11} \Bigg) \\
 							  &= \frac{1}{\sqrt{2}} \Bigg( \ket{00}  + \ket{11} \Bigg) \\
 							  &= \psi.
\end{align*}

\section*{edx Problem 5}
Let
\[
\ket{\psi} = \frac{1}{\sqrt{2}} (\ket{01} + \ket{10}).
\]
\begin{align*}
H^{\otimes 2} \frac{1}{\sqrt{2}} \ket{01} &= \frac{1}{\sqrt{2}} \ket{+-} \\
                                        						&= \frac{1}{\sqrt{2}} \Bigg( \frac{1}{\sqrt{2}} \ket{0} + \frac{1}{\sqrt{2}} \ket{1} \Bigg)\Bigg( \frac{1}{\sqrt{2}} \ket{0} - \frac{1}{\sqrt{2}} \ket{1} \Bigg) \\
                                        						&= \frac{1}{2\sqrt{2}} \Bigg( \ket{00} - \ket{01} + \ket{10} - \ket{11} \Bigg).
\end{align*}
\begin{align*}
H^{\otimes 2} \frac{1}{\sqrt{2}} \ket{10} &= \frac{1}{\sqrt{2}} \ket{-+} \\
                                        						&= \frac{1}{\sqrt{2}} \Bigg( \frac{1}{\sqrt{2}} \ket{0} - \frac{1}{\sqrt{2}} \ket{1} \Bigg)\Bigg( \frac{1}{\sqrt{2}} \ket{0} + \frac{1}{\sqrt{2}} \ket{1} \Bigg) \\
                                        						&= \frac{1}{2\sqrt{2}} \Bigg( \ket{00} + \ket{01} - \ket{10} - \ket{11} \Bigg).
\end{align*}
Now we see that $H^{\otimes 2} \ket{\psi}$ is
\begin{align*}
\frac{1}{2\sqrt{2}} \Bigg( \ket{00} - \ket{01} + \ket{10} - \ket{11} \Bigg) + \frac{1}{2\sqrt{2}} \Bigg( \ket{00} + \ket{01} - \ket{10} - \ket{11} \Bigg) &= \frac{1}{2\sqrt{2}} (2\ket{00} - 2\ket{11}) \\
     &= \frac{1}{\sqrt{2}} (\ket{00} - \ket{11}).
\end{align*}

\section*{edx Problem 6}
Yes. Both may yield $\ket{00}$ with probability $\frac{1}{2}$ and $\ket{11}$ with probability $\frac{1}{2}$.

\section*{edx Problem 7} 
Apply circuit A and then D.

\section*{edx Problem 8}
\[
A = \begin{pmatrix}
1 & 0 & 0 & 0 \\
0 & 0 & 1 & 0 \\
0 & 1 & 0 & 0 \\
0 & 0 & 0 & 1 
\end{pmatrix}.
\]

\section*{edx Problem 9}
Suppose
\[
H^{\otimes 3} \ket{\psi} = \frac{1}{\sqrt{2}} (\ket{000} + \ket{111}).
\]
Since $H^{\otimes 3}$ is reversible, if we apply it again to $H^{\otimes 3} \ket{\psi}$, we will obtain $\ket{\psi}$. Let us calculate that ket by ket:
\begin{align*}
H^{\otimes 3}\frac{1}{\sqrt{2}} \ket{000} &= \frac{1}{\sqrt{2}} \ket{+++} \\
		&= \frac{1}{4} \Bigg( \ket{0} + \ket{1} \Bigg)^3 \\
		&= \frac{1}{4} \Bigg( \ket{0} + \ket{1} \Bigg) \Bigg( \ket{00} + \ket{01} + \ket{10} + \ket{11} \Bigg) \\
		&= \frac{1}{4} \Bigg( \ket{000} + \ket{001} + \ket{010} + \ket{011} + \ket{100} + \ket{101} + \ket{110} + \ket{111} \Bigg).
\end{align*}
\begin{align*}
H^{\otimes 3}\frac{1}{\sqrt{2}} \ket{111} &= \frac{1}{\sqrt{2}} \ket{---} \\
		&= \frac{1}{4} \Bigg( \ket{0} - \ket{1} \Bigg)^3 \\
		&= \frac{1}{4} \Bigg( \ket{0} - \ket{1} \Bigg) \Bigg( \ket{00} - \ket{01} - \ket{10} + \ket{11} \Bigg) \\
		&= \frac{1}{4} \Bigg( \ket{000} - \ket{001} - \ket{010} + \ket{011} - \ket{100} + \ket{101} + \ket{110} - \ket{111} \Bigg).
\end{align*}
Now
\begin{align*}
\ket{\psi} &= H^{\otimes 3} H^{\otimes 3}  \ket{\psi} \\
                &= H^{\otimes 3} \frac{1}{\sqrt{2}} \ket{000} + H^{\otimes 3} \frac{1}{\sqrt{2}} \ket{11} \\
                &= \frac{1}{4} (2\ket{000} + 2\ket{011} + 2\ket{101} + 2\ket{110}) \\
                &= \frac{1}{2} (\ket{000} + \ket{011} + \ket{101} + \ket{110}).
\end{align*}

\section*{edx Problem 10}
(a)
\[
\frac{1}{2^{n - 1}}.
\]
(b) We see a uniformly random string $y \in \{ 0, 1 \}^n$.

\section*{edx Problem 11}
\[
\frac{1}{\sqrt{2^n}} \sum_{x \in \{ 0, 1 \}^n} (-1)^{x \cdot f(x)} \ket{x} \ket{f(x)}.
\]

\section*{edx Problem 12}
\[
\frac{1}{\sqrt{2^n}} \sum_{x} \ket{x} \ket{0101}. 
\]

\section*{edx Problem 13}
1111.

\section*{edx Problem 14}
Suppose Alice starts with two qubits in the Bell state $\frac{1}{\sqrt{2}} \ket{00} + \frac{1}{\sqrt{2}} \ket{11}$ and teleports these qubits to Bob by applying the quantum teleportation protocol to each qubit separately.

As we are speaking about teleportation, Bob sees the same state and receives exactly 2 bits of information as there is only 4 Bell states.

\section*{QCE 7.1}
\[
\vec{\sigma} = \sigma_x \hat{x} + \sigma_y \hat{y} + \sigma_z \hat{z}, \qquad
\vec{n} = \sin \theta \cos \phi \hat{x} + \sin \theta \sin \phi \hat{y} + \cos \theta \hat{z}.
\]
Now
\begin{align*}
\vec{\sigma} \cdot \vec{n} &= (\sigma_x \hat{x} + \sigma_y \hat{y} + \sigma_z \hat{z})(\sin \theta \cos \phi \hat{x} + \sin \theta \sin \phi \hat{y} + \cos \theta \hat{z}) \\
            							   &= 
\end{align*}

\section*{QCE 7.2}
Let 
\[
\ket{\psi} = \frac{\ket{0}\ket{1} - \ket{1}\ket{0}}{\sqrt{2}}.
\]
Now we have
\[ 
\ket{0} = \frac{\ket{+} + i\ket{-}}{\sqrt{2}}, \; \ket{1} = \frac{\ket{+} - i\ket{-}}{\sqrt{2}},
\]
so 
\begin{align*}
\ket{\psi} &= \frac{\ket{0}\ket{1} - \ket{1}\ket{0}}{\sqrt{2}} \\
                &= \frac{1}{2\sqrt{2}} \Bigg( (\ket{+} + i\ket{-}) ( \ket{+} - i\ket{-} ) - (\ket{+} - i\ket{-}) ( \ket{+} + i\ket{-} ) \Bigg) \\
                &= \frac{1}{2\sqrt{2}} \Bigg( \ket{++} - i\ket{+-} + i \ket{-+} +\ket{--} + \ket{++} + i\ket{+-} -i \ket{-+} + \ket{--}\Bigg) \\
                &= \frac{1}{2\sqrt{2}} ( 2\ket{++}  + 2\ket{--}) \\
                &= \frac{1}{\sqrt{2}} (\ket{++} + \ket{--}).
\end{align*}

\section*{QCE 7.3}
Let
\[
\ket{\beta_{00}} = \frac{\ket{00} + \ket{11}}{\sqrt{2}} 
= \frac{1}{\sqrt{2}} 
\begin{pmatrix}
1 \\
0 \\
0 \\
1
\end{pmatrix},
 \; 
 \ket{\beta_{01}} = \frac{\ket{01} + \ket{10}}{\sqrt{2}} 
 =
\frac{1}{\sqrt{2}} 
 \begin{pmatrix}
 0 \\
 1 \\
 1 \\
 0
 \end{pmatrix}.
\]
Also we know that
\[
Z = \begin{pmatrix}
1 & 0 \\
0 & - 1
\end{pmatrix},
\]
so 
\begin{align*}
Z \otimes Z &= \begin{pmatrix}
1 & 0 \\
0 & - 1
\end{pmatrix}
\otimes
\begin{pmatrix}
1 & 0 \\
0 & - 1
\end{pmatrix} \\
&= 
\begin{pmatrix}
1 &  0  & 0   & 0 \\
0 & -1 & 0   & 0 \\
0 &  0  & -1 & 0 \\
0 &  0  &  0  & 1
\end{pmatrix}.
\end{align*}
Now
\begin{align*}
Z \otimes Z \ket{\beta_{00}} &= \begin{pmatrix}
1 &  0  & 0   & 0 \\
0 & -1 & 0   & 0 \\
0 &  0  & -1 & 0 \\
0 &  0  &  0  & 1
\end{pmatrix}
\frac{1}{\sqrt{2}} 
\begin{pmatrix}
1 \\
0 \\
0 \\
1
\end{pmatrix} \\
&= \frac{1}{\sqrt{2}}
\begin{pmatrix}
1 \\
0 \\
0 \\
1
\end{pmatrix} \\
&= \beta_{00} \\
&= (-1)^y \ket{\beta_{xy}},
\end{align*}
where $y = 0$ and $x = 0$.
Also
\begin{align*}
Z \otimes Z \ket{\beta_{01}} &= \begin{pmatrix}
1 &  0  & 0   & 0 \\
0 & -1 & 0   & 0 \\
0 &  0  & -1 & 0 \\
0 &  0  &  0  & 1
\end{pmatrix}
\frac{1}{\sqrt{2}} 
\begin{pmatrix}
0 \\
1 \\
1 \\
0
\end{pmatrix} \\
&= \frac{1}{\sqrt{2}}
\begin{pmatrix}
0 \\
-1 \\
-1 \\
0
\end{pmatrix} \\
&= -\beta_{01} \\
&= (-1)^y \ket{\beta_{xy}},
\end{align*}
where $y = 1$ and $x = 0$.

\section*{QCE 7.4}
Show that $X \otimes X \ket{\beta_{xy}} = (-1)^x \ket{\beta_{xy}}$. From the book we know that 
\[
\ket{\beta_{xy}} = \frac{\ket{0y} + (-1)^x\ket{1\bar{y}}}{\sqrt{2}} 
= 
\frac{1}{\sqrt{2}}
\begin{pmatrix}
\bar{y} \\
y \\
(-1)^x y \\
(-1)^x \bar{y}
\end{pmatrix}.
\]
Also 
\begin{align*}
X \otimes X &= 
\begin{pmatrix}
0 & 1 \\
1 & 0
\end{pmatrix}
\otimes 
\begin{pmatrix}
0 & 1 \\
1 & 0
\end{pmatrix} \\
&= \begin{pmatrix}
0 & 0 & 0 & 1 \\
0 & 0 & 1 & 0 \\
0 & 1 & 0 & 0 \\
1 & 0 & 0 & 0
\end{pmatrix},
\end{align*}
so
\begin{align*}
X \otimes X \ket{\beta_{xy}} &= 
		\frac{1}{\sqrt{2}}
		\begin{pmatrix}
        0 & 0 & 0 & 1 \\
        0 & 0 & 1 & 0 \\
        0 & 1 & 0 & 0 \\
        1 & 0 & 0 & 0
		\end{pmatrix}
		\begin{pmatrix}
        \bar{y} \\
        y \\
        (-1)^x y \\
        (-1)^x \bar{y} \\
\end{pmatrix} \\
&= \frac{1}{\sqrt{2}}
\begin{pmatrix}
        (-1)^x \bar{y} \\
        (-1)^x y \\
        y \\
        \bar{y} \\
\end{pmatrix} \\
&= \frac{(-1)^x \ket{0y} + \ket{1\bar{y}}}{\sqrt{2}} \\
&= (-1)^x \beta_{xy}.
\end{align*}

\section*{QCE 7.5}
Show that $Y \otimes Y \ket{\beta_{xy}} = (-1)^{x + y} \ket{\beta_{xy}}$.
From the book we know that 
\[
\ket{\beta_{xy}} = \frac{\ket{0y} + (-1)^x\ket{1\bar{y}}}{\sqrt{2}} 
= 
\frac{1}{\sqrt{2}}
\begin{pmatrix}
\bar{y} \\
y \\
(-1)^x y \\
(-1)^x \bar{y}
\end{pmatrix}.
\]
Also 
\begin{align*}
Y \otimes Y &= 
\begin{pmatrix}
0 & -i \\
i & 0
\end{pmatrix}
\otimes 
\begin{pmatrix}
0 & -i \\
i & 0
\end{pmatrix} \\
&= \begin{pmatrix}
0 & 0 & 0 & -1 \\
0 & 0 & 1 & 0 \\
0 & 1 & 0 & 0 \\
-1 & 0 & 0 & 0
\end{pmatrix},
\end{align*}
so
\begin{align*}
Y \otimes Y \ket{\beta_{xy}} &= 
		\frac{1}{\sqrt{2}}
		\begin{pmatrix}
        0 & 0 & 0 & -1 \\
        0 & 0 & 1 & 0 \\
        0 & 1 & 0 & 0 \\
        -1 & 0 & 0 & 0
		\end{pmatrix}
		\begin{pmatrix}
        \bar{y} \\
        y \\
        (-1)^x y \\
        (-1)^x \bar{y} \\
\end{pmatrix} \\
&= \frac{1}{\sqrt{2}}
\begin{pmatrix}
        -(-1)^x \bar{y} \\
        (-1)^x y \\
        y \\
        -\bar{y} \\
\end{pmatrix} \\
&= (-1)^{x + \bar{y}} \ket{\beta_{xy}}.
\end{align*}
The book has a typo...

\section*{QCE 7.6}
Show that $X \otimes X$ commutes with $Z \otimes Z$. They commute if and only if $X \otimes X Z \otimes Z = X \otimes X Z \otimes Z$:
\begin{align*}
X \otimes X &= \begin{pmatrix}
0 & 1 \\
1 & 0
\end{pmatrix}^{\otimes 2} \\
&= \begin{pmatrix}
0 & 0 & 0 & 1 \\
0 & 0 & 1 & 0 \\
0 & 1 & 0 & 0 \\
1 & 0 & 0 & 0 
\end{pmatrix}.
\end{align*}
\begin{align*}
Z \otimes Z &= \begin{pmatrix}
1 & 0 \\
0 & -1
\end{pmatrix}^{\otimes 2} \\
&= 
\begin{pmatrix}
1 & 0   & 0  & 0 \\
0 & -1 & 0  & 0 \\
0 & 0   & -1 & 0 \\
0 & 0   & 0   & 1
\end{pmatrix}.
\end{align*}
Now
\[
\begin{pmatrix}
0 & 0 & 0 & 1 \\
0 & 0 & 1 & 0 \\
0 & 1 & 0 & 0 \\
1 & 0 & 0 & 0 
\end{pmatrix}
\begin{pmatrix}
1 & 0   & 0  & 0 \\
0 & -1 & 0  & 0 \\
0 & 0   & -1 & 0 \\
0 & 0   & 0   & 1
\end{pmatrix}
=
\begin{pmatrix}
0 & 0 & 0 & 1 \\
0 & 0 & -1 & 0 \\
0 & -1 & 0 & 0 \\
1 & 0 & 0 & 0 
\end{pmatrix}.
\]
Also 
\[
\begin{pmatrix}
1 & 0   & 0  & 0 \\
0 & -1 & 0  & 0 \\
0 & 0   & -1 & 0 \\
0 & 0   & 0   & 1
\end{pmatrix}
\begin{pmatrix}
0 & 0 & 0 & 1 \\
0 & 0 & 1 & 0 \\
0 & 1 & 0 & 0 \\
1 & 0 & 0 & 0 
\end{pmatrix}
=
\begin{pmatrix}
0 & 0 & 0 & 1 \\
0 & 0 & -1 & 0 \\
0 & -1 & 0 & 0 \\
1 & 0 & 0 & 0 
\end{pmatrix},
\]
which proves that the two matrices commute.

\section*{QCE 7.7}
Consider the eigenvectors in Example 7.4. Show that $[H_I, \vec{\sigma_A} \cdot \vec{\sigma_B}] = 0$, and hence show that the eigenvectors of the Hamiltonian are eigenvectors of the $\vec{\sigma_A} \cdot \vec{\sigma_B}$ operator. In particular, show that $\vec{\sigma_A} \cdot \vec{\sigma_B} \ket{\phi_i} = \ket{\phi_i}$ for $i = 1, 2, 3$ and 
$\vec{\sigma_A} \cdot \vec{\sigma_B} \ket{\phi_4} = -3 \ket{\phi_4}$.

From the Example 7.4 we have that
\[
H_I = 
\frac{\mu^2}{r^3}
\begin{pmatrix}
-2 & 0 & 0 & 0 \\
0 & 2 & 2 & 0 \\
0 & 2 & 2 & 0 \\
0 & 0 & 0 & -2
\end{pmatrix},
\;
\vec{\sigma_A} \cdot \vec{\sigma_B} = 
\begin{pmatrix}
1 & 0 & 0 & 0 \\
0 & -1 & 2 & 0 \\
0 & 2 & -1 & 0 \\
0 & 0 & 0 & 1
\end{pmatrix}.
\]
Condition $[H_I, \vec{\sigma_A} \cdot \vec{\sigma_B}] = 0$ means that the two matrices commute and that is the case as a routine calculation may show. Wolframalpha tells me that the eigenvectors of $H_I$ are
\[
\begin{pmatrix}
0 \\
-1 \\
-1 \\
0
\end{pmatrix}, \qquad
\begin{pmatrix}
0 \\
0 \\
0 \\
1
\end{pmatrix}, \qquad
\begin{pmatrix}
1 \\
0 \\
0 \\
0
\end{pmatrix}, \qquad
\begin{pmatrix}
0 \\
-1 \\
1 \\
0
\end{pmatrix},
\]
and that is not in accord with the eigenvectors of $\vec{\sigma_A} \cdot \vec{\sigma_B}$, thank you, book. :(

Now
\begin{align*}
\vec{\sigma_A} \cdot \vec{\sigma_B} \ket{\phi_1} &= 
\begin{pmatrix}
1 & 0 & 0 & 0 \\
0 & -1 & 2 & 0 \\
0 & 2 & -1 & 0 \\
0 & 0 & 0 & 1
\end{pmatrix} 
\frac{1}{\sqrt{2}}
\begin{pmatrix}
0 \\
1 \\
1 \\
0
\end{pmatrix} \\
   &= 
   \frac{1}{\sqrt{2}}
\begin{pmatrix}
0 \\
1 \\
1 \\
0
\end{pmatrix} \\
&= \ket{\phi_1},
\end{align*}
\begin{align*}
\vec{\sigma_A} \cdot \vec{\sigma_B} \ket{\phi_2} &=
\begin{pmatrix}
1 & 0 & 0 & 0 \\
0 & -1 & 2 & 0 \\
0 & 2 & -1 & 0 \\
0 & 0 & 0 & 1
\end{pmatrix} 
\begin{pmatrix}
0 \\
0 \\
0 \\
1
\end{pmatrix} \\
&= 
\begin{pmatrix}
0 \\
0 \\
0 \\
1 
\end{pmatrix} \\
&= \ket{\phi_2},
\end{align*}
\begin{align*}
\vec{\sigma_A} \cdot \vec{\sigma_B} \ket{\phi_3} &= 
\begin{pmatrix}
1 & 0 & 0 & 0 \\
0 & -1 & 2 & 0 \\
0 & 2 & -1 & 0 \\
0 & 0 & 0 & 1
\end{pmatrix} 
\begin{pmatrix}
1 \\
0 \\
0 \\
0
\end{pmatrix} \\
&= 
\begin{pmatrix}
1 \\
0 \\
0 \\
0
\end{pmatrix} \\
&= \ket{\phi_3},
\end{align*}
\begin{align*}
\vec{\sigma_A} \cdot \vec{\sigma_B} \ket{\phi_4} &= 
\begin{pmatrix}
1 & 0 & 0 & 0 \\
0 & -1 & 2 & 0 \\
0 & 2 & -1 & 0 \\
0 & 0 & 0 & 1
\end{pmatrix}
\frac{1}{\sqrt{2}}
\begin{pmatrix}
0 \\
1 \\
-1 \\
0
\end{pmatrix}  \\
&= 
\frac{1}{\sqrt{2}}
\begin{pmatrix}
0 \\
-3 \\
3 \\
0
\end{pmatrix} \\
&= -3 \ket{\phi_4}.
\end{align*}

\section*{QCE 7,8}
Is the state $X \otimes Z \ket{\beta_{00}}$ entangled?
\begin{align*}
X \otimes Z \ket{00} &= 
\begin{pmatrix}
0 & 1 \\
1 & 0 
\end{pmatrix}
\otimes
\begin{pmatrix}
1 & 0 \\
0 & - 1
\end{pmatrix}
\frac{1}{\sqrt{2}}
\begin{pmatrix}
1 \\ 0 \\ 0 \\ 1
\end{pmatrix} \\
&= 
\frac{1}{\sqrt{2}}
\begin{pmatrix}
0 & 0   & 1 & 0 \\
0 & 0   & 0 & -1 \\
1 & 0   & 0 & 0 \\
0 & -1 & 0 & 0 
\end{pmatrix}
\begin{pmatrix}
1 \\ 0 \\ 0 \\ 1
\end{pmatrix} \\
&= \frac{1}{\sqrt{2}}
\begin{pmatrix}
0 \\ -1 \\ 1 \\ 0
\end{pmatrix} \\
&= \frac{-\ket{01} + \ket{10}}{\sqrt{2}} \\
&= -\ket{\beta_{11}},
\end{align*}
so the state is entangled but is not a Bell state.

\end{document}