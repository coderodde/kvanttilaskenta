\documentclass[10pt]{article}

\usepackage[english]{babel}
\usepackage[utf8x]{inputenc}
\usepackage{amsmath}
\usepackage{amssymb}
\usepackage{amsfonts}
\usepackage{graphicx}
\usepackage[ruled,linesnumbered,noend]{algorithm2e}
\usepackage{empheq}
\usepackage{float}
\usepackage{enumitem}
\usepackage{tikz}
\usepackage[colorlinks=true,urlcolor=blue]{hyperref}
\usepackage{braket}
\usepackage{gensymb}

\title{Kvanttilaskenta, kevät 2015 -- Viikko 6}
\author{Rodion ``rodde'' Efremov}

\begin{document}
 \maketitle
 
 \section*{edx Problem 1}
 \[
 \ket{\psi} = \alpha \ket{0} + \beta \ket{1} 
 = 
 \begin{pmatrix}
 \alpha \\
 \beta
 \end{pmatrix}.
 \]
 \[
 \ket{-} = 
 \begin{pmatrix}
\frac{1}{\sqrt{2}} \\
-\frac{1}{\sqrt{2}} 
 \end{pmatrix}.
 \]
\[
CNOT = 
\begin{pmatrix}
1 & 0 & 0 & 0 \\
0 & 1 & 0 & 0 \\
0 & 0 & 0 & 1 \\
0 & 0 & 1 & 0 \\
\end{pmatrix}.
\]
Now
\begin{align*}
\begin{pmatrix}
1 & 0 & 0 & 0 \\
0 & 1 & 0 & 0 \\
0 & 0 & 0 & 1 \\
0 & 0 & 1 & 0 \\
\end{pmatrix}
\begin{pmatrix}
\alpha \\ \beta \\ \frac{1}{\sqrt{2}} \\ -\frac{1}{\sqrt{2}}
\end{pmatrix} 
&= 
\begin{pmatrix}
\alpha \\
\beta \\
-\frac{1}{\sqrt{2}} \\
\frac{1}{\sqrt{2}}
\end{pmatrix}.
\end{align*}
(a) The resulting state is not entangled. (b) The probability in question is 1. (c) The state of the first qubit does not change.

\section*{edx Problem 2}
Is the ``phase inversion'' unitary its own inverse? Answer: Yes.

\section*{edx Problem 3}
Is the ``inversion about the mean'' unitary its own inverse? Answer: Yes.

\section*{edx Problem 4}

\section*{edx Problem 5}

\section*{edx Problem 6}
Now, consider the case where $\frac{N}{4}$ elements are marked instead of just one. If we run one iteration of Grover's algorithm and measure, what is the probability that we see a marked element?

It appears that the probability in question is $\binom{N}{4}^{-1}$.

\section*{edx Problem 7}
Which of the following observables correspond to a standard basis measurements Answer: $I$.

\section*{edx Problem 8}
Which of the following observables correspond to a sign basis measurement? Answer: $H$.

\section*{edx Problem 9}
Suppose we measure a qubit $\alpha \ket{0} + \beta \ket{1}$ with respect to the observable $I = \begin{pmatrix}
1 & 0 \\ 0 & 1
\end{pmatrix}.$ What is the outcome of the measurement? Answer: (1) It is 0 with probability $|\alpha|^2$ with the new state $\ket{0}$, and 1 with probability $|\beta|^2$ with the new state $\ket{1}$.

\section*{edx Problem 10}

\section*{edx Problem 11}

\section*{edx Problem 12}
The Hamiltonian is 
\[
\begin{pmatrix}
1 & 4 \\ 4 & 1
\end{pmatrix}.
\]
The states of definite energy are
\[
\begin{pmatrix}
1 \\ 1
\end{pmatrix},
\begin{pmatrix}
-1 \\ 1
\end{pmatrix},
\]
and the corresponding energies are 5 and -3.

\section*{edx Problem 13}

\end{document}