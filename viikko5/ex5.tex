\documentclass[10pt]{article}

\usepackage[english]{babel}
\usepackage[utf8x]{inputenc}
\usepackage{amsmath}
\usepackage{amssymb}
\usepackage{amsfonts}
\usepackage{graphicx}
\usepackage[ruled,linesnumbered,noend]{algorithm2e}
\usepackage{empheq}
\usepackage{float}
\usepackage{enumitem}
\usepackage{tikz}
\usepackage[colorlinks=true,urlcolor=blue]{hyperref}
\usepackage{braket}
\usepackage{gensymb}

\title{Kvanttilaskenta, kevät 2015 -- Viikko 5}
\author{Rodion ``rodde'' Efremov}

\begin{document}
 \maketitle

\section*{edx Problem 1}
We have
\begin{align*}
QFT_M &= 
\frac{1}{\sqrt{6}}
\begin{pmatrix}
1 & 1 & 1 & 1 & 1 & 1 \\
1 & \omega & \omega^{2} & \omega^{3} & \omega^{4} & \omega^{5} \\
1 & \omega^{2} & \omega^{4} & \omega^{6} & \omega^{8} & \omega^{10} \\
1 & \omega^{3} & \omega^{6} & \omega^{9} & \omega^{12} & \omega^{15} \\
1 & \omega^{4} & \omega^{8} & \omega^{12} & \omega^{16} & \omega^{20} \\
1 & \omega^{5} & \omega^{10} & \omega^{15} & \omega^{20} & \omega^{25} 
\end{pmatrix} \\
&= 
\frac{1}{\sqrt{6}}
\begin{pmatrix}
1 & 1 & 1 & 1 & 1 & 1 \\
1 & \omega & \omega^{2} & \omega^{3} & \omega^{4} & \omega^{5} \\
1 & \omega^{2} & \omega^{4} & 1 & \omega^{2} & \omega^{4} \\
1 & \omega^{3} & 1 & \omega^{3} & 1 & \omega^{3} \\
1 & \omega^{4} & \omega^{2} & 1 & \omega^{4} & \omega^{2} \\
1 & \omega^{5} & \omega^{4} & \omega^{3} & \omega^{2} & \omega^{1} \\
\end{pmatrix}
\end{align*}
Now $\omega = e^{2\pi i j / M}$, for $j = 0, 1, \dots, M - 1$.

\section*{edx Problem 2}
What is $QFT_6$ of $\frac{1}{\sqrt{2}} (\ket{0} + \ket{3})$? 

Now the matrix representation of the input state is
\[
\ket{\psi} = \frac{1}{\sqrt{2}} 
\begin{pmatrix}
1 \\ 0 \\ 0 \\ 1 \\ 0 \\ 0
\end{pmatrix},
\]
so the QFT of $\ket{\psi}$ is
\[
\frac{1}{\sqrt{12}}
\begin{pmatrix}
2 \\ 
1  + \omega^{3} \\
2 \\
1 + \omega^{3} \\
2 \\
1 + \omega^{3}
\end{pmatrix},
\]
where $\omega^{3} = e^{2\pi i \times 3 / 6} = e^{\pi i} = \cos \pi + i \sin \pi = -1$, so
\[
\ket{\psi} = \frac{1}{\sqrt{12}}
\begin{pmatrix}
2 \\ 0 \\2 \\ 0 \\2 \\ 0 
\end{pmatrix}.
\]
The probability of measuring $\ket{0}, \ket{2}, \ket{4}$ is $\frac{1}{3}$ each and other probabilities are 0.

\section*{edx Problem 3}
What is $QFT_6$ of $\frac{1}{\sqrt{2}} (\ket{1} + \ket{4})$? Now we have a state
\[
\ket{\psi} = \frac{1}{\sqrt{2}}
\begin{pmatrix}
0 \\
1 \\
0 \\
0 \\
1 \\
0
\end{pmatrix},
\]
so the result is
\[
\frac{1}{\sqrt{12}}
\begin{pmatrix}
2 \\
\omega + \omega^{4} \\
2\omega^{2} \\
1 + \omega^{3} \\
2\omega^{4} \\
\omega^{5} + \omega^{2}
\end{pmatrix}
= 
\frac{1}{\sqrt{12}}
\begin{pmatrix}
2 \\ 0 \\ 2 \omega^{2} \\ 0 \\ 2 \omega^4 \\ 0
\end{pmatrix}.
\]
Now it is easy to see that the probability of measuring $\ket{0}, \ket{2}, \ket{4}$ is $\frac{1}{3}$ each and zero probability of seeing other.

\section*{edx Problem 4}
What is $QFT_6$ of $\frac{1}{\sqrt{3}} ( \ket{0} + \ket{2} + \ket{4} )$? Now the input state is
\[
\ket{\psi} = 
\frac{1}{\sqrt{3}}
\begin{pmatrix}
1 \\ 0 \\ 1 \\ 0 \\ 1 \\ 0 \
\end{pmatrix},
\]
so that the result state is
\[
\frac{1}{\sqrt{18}}
\begin{pmatrix}
3 \\
1 + \omega^2 + \omega^4 \\
1 + \omega^4 + \omega^2 \\
3 \\
1 + \omega^2 + \omega^4 \\
1 + \omega^4 + \omega^2 
\end{pmatrix}
= 
\frac{1}{\sqrt{18}}
\begin{pmatrix}
3 \\
0 \\
0 \\
3 \\
0 \\
0
\end{pmatrix},
\]
 so the probability of measuring $\ket{0}$ or $\ket{3}$ is $\frac{1}{2}$ each and the other states has probability 0.

\section*{edx Problem 5}
What is $QFT_6$ of $\frac{1}{\sqrt{3}} ( \ket{1} + \ket{3} + \ket{5} )$? The input state is
\[
\frac{1}{\sqrt{3}}
\begin{pmatrix}
0 \\ 1 \\ 0 \\ 1 \\ 0 \\ 1 
\end{pmatrix},
\] 
so the result state is 
\[
\frac{1}{\sqrt{18}}
\begin{pmatrix}
3 \\
\omega + \omega^3 + \omega^5 \\
\omega^2 + 1 + \omega^4 \\
3 \omega^3 \\
\omega^4 + 1 + \omega^2 \\
\omega^5 + \omega^3 + \omega
\end{pmatrix}
= 
\frac{1}{\sqrt{18}}
\begin{pmatrix}
3 \\ 0 \\ 0 \\ -3 \\ 0 \\ 0 \\
\end{pmatrix}.
\]
 The probability of measuring $\ket{0}$ or $\ket{3}$ is $\frac{1}{2}$ each, and other probabilities are zero.
 
\section*{edx Problem 7}
Consider the periodic superposition $\ket{a} = \sqrt{\frac{k}{M}} \sum_{j = 0}^{M/k - 1} \ket{jk}$. Let $\beta = \sum_j \beta_j \ket{j}$ be its $QFT_M$.

(a) Derive an expression for $\beta_j$. The desired expression is
\[
\frac{\sqrt{k}}{M} \sum_{l = 0}^{M/k - 1} e^{2\pi j l k i/M}.
\] 

(b) If $j$ is a multiple of $M/k$, what is the value of $\beta_j$? So we have that $j = aM / k$ for some nonnegative integer $a$. Now
\begin{align*}
\frac{\sqrt{k}}{M} \sum_{l = 0}^{M/k - 1} e^{2\pi j l k i/M} &= \frac{\sqrt{k}}{M} \sum_{l = 0}^{M/k - 1} e^{2\pi a l  i} \\
&= \frac{\sqrt{k}}{M} \sum_{l = 0}^{M/k - 1} 1 \\
&= \frac{\sqrt{k}}{M} M/k \\
&= \frac{1}{\sqrt{k}}.
\end{align*}

(c) If $j$ is not a multiple of $M/k$, what is the value of $\beta_j$? Now suppose $j = aM/k + r$ for some nonnegative integer $a$ and an integer $r \in [1, a)$. We have
\begin{align*}
\frac{\sqrt{k}}{M} \sum_{l = 0}^{M/k - 1} e^{2\pi j l k i/M} &= \frac{\sqrt{k}}{M} \sum_{l = 0}^{M/k - 1} e^{2\pi l k i ( aM / k + r ) / M} \\
&= \frac{\sqrt{k}}{M} \sum_{l = 0}^{M/k - 1} e^{2\pi l k i ( aM / k ) / M + 2\pi l k i r / M }  \\
&= \frac{\sqrt{k}}{M} \sum_{l = 0}^{M/k - 1} e^{2 \pi l i a} e^{2 i \pi l k r / M} \\
&= \frac{\sqrt{k}}{M} \sum_{l = 0}^{M/k - 1} e^{2 i \pi l k r / M} \\
&= 
\end{align*}

\section*{QCE 7.1}
\[
\vec{\sigma} = \sigma_x \hat{x} + \sigma_y \hat{y} + \sigma_z \hat{z}, \qquad
\vec{n} = \sin \theta \cos \phi \hat{x} + \sin \theta \sin \phi \hat{y} + \cos \theta \hat{z}.
\]
Now
\begin{align*}
\vec{\sigma} \cdot \vec{n} &= (\sigma_x \hat{x} + \sigma_y \hat{y} + \sigma_z \hat{z})(\sin \theta \cos \phi \hat{x} + \sin \theta \sin \phi \hat{y} + \cos \theta \hat{z}) \\
            							   &= 
\end{align*}

\section*{QCE 7.2}
Let 
\[
\ket{\psi} = \frac{\ket{0}\ket{1} - \ket{1}\ket{0}}{\sqrt{2}}.
\]
Now we have
\[ 
\ket{0} = \frac{\ket{+} + i\ket{-}}{\sqrt{2}}, \; \ket{1} = \frac{\ket{+} - i\ket{-}}{\sqrt{2}},
\]
so 
\begin{align*}
\ket{\psi} &= \frac{\ket{0}\ket{1} - \ket{1}\ket{0}}{\sqrt{2}} \\
                &= \frac{1}{2\sqrt{2}} \Bigg( (\ket{+} + i\ket{-}) ( \ket{+} - i\ket{-} ) - (\ket{+} - i\ket{-}) ( \ket{+} + i\ket{-} ) \Bigg) \\
                &= \frac{1}{2\sqrt{2}} \Bigg( \ket{++} - i\ket{+-} + i \ket{-+} +\ket{--} + \ket{++} + i\ket{+-} -i \ket{-+} + \ket{--}\Bigg) \\
                &= \frac{1}{2\sqrt{2}} ( 2\ket{++}  + 2\ket{--}) \\
                &= \frac{1}{\sqrt{2}} (\ket{++} + \ket{--}).
\end{align*}

\section*{QCE 7.3}
Let
\[
\ket{\beta_{00}} = \frac{\ket{00} + \ket{11}}{\sqrt{2}} 
= \frac{1}{\sqrt{2}} 
\begin{pmatrix}
1 \\
0 \\
0 \\
1
\end{pmatrix},
 \; 
 \ket{\beta_{01}} = \frac{\ket{01} + \ket{10}}{\sqrt{2}} 
 =
\frac{1}{\sqrt{2}} 
 \begin{pmatrix}
 0 \\
 1 \\
 1 \\
 0
 \end{pmatrix}.
\]
Also we know that
\[
Z = \begin{pmatrix}
1 & 0 \\
0 & - 1
\end{pmatrix},
\]
so 
\begin{align*}
Z \otimes Z &= \begin{pmatrix}
1 & 0 \\
0 & - 1
\end{pmatrix}
\otimes
\begin{pmatrix}
1 & 0 \\
0 & - 1
\end{pmatrix} \\
&= 
\begin{pmatrix}
1 &  0  & 0   & 0 \\
0 & -1 & 0   & 0 \\
0 &  0  & -1 & 0 \\
0 &  0  &  0  & 1
\end{pmatrix}.
\end{align*}
Now
\begin{align*}
Z \otimes Z \ket{\beta_{00}} &= \begin{pmatrix}
1 &  0  & 0   & 0 \\
0 & -1 & 0   & 0 \\
0 &  0  & -1 & 0 \\
0 &  0  &  0  & 1
\end{pmatrix}
\frac{1}{\sqrt{2}} 
\begin{pmatrix}
1 \\
0 \\
0 \\
1
\end{pmatrix} \\
&= \frac{1}{\sqrt{2}}
\begin{pmatrix}
1 \\
0 \\
0 \\
1
\end{pmatrix} \\
&= \beta_{00} \\
&= (-1)^y \ket{\beta_{xy}},
\end{align*}
where $y = 0$ and $x = 0$.
Also
\begin{align*}
Z \otimes Z \ket{\beta_{01}} &= \begin{pmatrix}
1 &  0  & 0   & 0 \\
0 & -1 & 0   & 0 \\
0 &  0  & -1 & 0 \\
0 &  0  &  0  & 1
\end{pmatrix}
\frac{1}{\sqrt{2}} 
\begin{pmatrix}
0 \\
1 \\
1 \\
0
\end{pmatrix} \\
&= \frac{1}{\sqrt{2}}
\begin{pmatrix}
0 \\
-1 \\
-1 \\
0
\end{pmatrix} \\
&= -\beta_{01} \\
&= (-1)^y \ket{\beta_{xy}},
\end{align*}
where $y = 1$ and $x = 0$.

\section*{QCE 7.4}
Show that $X \otimes X \ket{\beta_{xy}} = (-1)^x \ket{\beta_{xy}}$. From the book we know that 
\[
\ket{\beta_{xy}} = \frac{\ket{0y} + (-1)^x\ket{1\bar{y}}}{\sqrt{2}} 
= 
\frac{1}{\sqrt{2}}
\begin{pmatrix}
\bar{y} \\
y \\
(-1)^x y \\
(-1)^x \bar{y}
\end{pmatrix}.
\]
Also 
\begin{align*}
X \otimes X &= 
\begin{pmatrix}
0 & 1 \\
1 & 0
\end{pmatrix}
\otimes 
\begin{pmatrix}
0 & 1 \\
1 & 0
\end{pmatrix} \\
&= \begin{pmatrix}
0 & 0 & 0 & 1 \\
0 & 0 & 1 & 0 \\
0 & 1 & 0 & 0 \\
1 & 0 & 0 & 0
\end{pmatrix},
\end{align*}
so
\begin{align*}
X \otimes X \ket{\beta_{xy}} &= 
		\frac{1}{\sqrt{2}}
		\begin{pmatrix}
        0 & 0 & 0 & 1 \\
        0 & 0 & 1 & 0 \\
        0 & 1 & 0 & 0 \\
        1 & 0 & 0 & 0
		\end{pmatrix}
		\begin{pmatrix}
        \bar{y} \\
        y \\
        (-1)^x y \\
        (-1)^x \bar{y} \\
\end{pmatrix} \\
&= \frac{1}{\sqrt{2}}
\begin{pmatrix}
        (-1)^x \bar{y} \\
        (-1)^x y \\
        y \\
        \bar{y} \\
\end{pmatrix} \\
&= \frac{(-1)^x \ket{0y} + \ket{1\bar{y}}}{\sqrt{2}} \\
&= (-1)^x \beta_{xy}.
\end{align*}

\section*{QCE 7.5}
Show that $Y \otimes Y \ket{\beta_{xy}} = (-1)^{x + y} \ket{\beta_{xy}}$.
From the book we know that 
\[
\ket{\beta_{xy}} = \frac{\ket{0y} + (-1)^x\ket{1\bar{y}}}{\sqrt{2}} 
= 
\frac{1}{\sqrt{2}}
\begin{pmatrix}
\bar{y} \\
y \\
(-1)^x y \\
(-1)^x \bar{y}
\end{pmatrix}.
\]
Also 
\begin{align*}
Y \otimes Y &= 
\begin{pmatrix}
0 & -i \\
i & 0
\end{pmatrix}
\otimes 
\begin{pmatrix}
0 & -i \\
i & 0
\end{pmatrix} \\
&= \begin{pmatrix}
0 & 0 & 0 & -1 \\
0 & 0 & 1 & 0 \\
0 & 1 & 0 & 0 \\
-1 & 0 & 0 & 0
\end{pmatrix},
\end{align*}
so
\begin{align*}
Y \otimes Y \ket{\beta_{xy}} &= 
		\frac{1}{\sqrt{2}}
		\begin{pmatrix}
        0 & 0 & 0 & -1 \\
        0 & 0 & 1 & 0 \\
        0 & 1 & 0 & 0 \\
        -1 & 0 & 0 & 0
		\end{pmatrix}
		\begin{pmatrix}
        \bar{y} \\
        y \\
        (-1)^x y \\
        (-1)^x \bar{y} \\
\end{pmatrix} \\
&= \frac{1}{\sqrt{2}}
\begin{pmatrix}
        -(-1)^x \bar{y} \\
        (-1)^x y \\
        y \\
        -\bar{y} \\
\end{pmatrix} \\
&= (-1)^{x + \bar{y}} \ket{\beta_{xy}}.
\end{align*}
The book has a typo as in the above formula in the term $x + y$ $y$ should be $\bar{y}$.

\section*{QCE 7.6}
Show that $X \otimes X$ commutes with $Z \otimes Z$. They commute if and only if $X \otimes X Z \otimes Z = X \otimes X Z \otimes Z$:
\begin{align*}
X \otimes X &= \begin{pmatrix}
0 & 1 \\
1 & 0
\end{pmatrix}^{\otimes 2} \\
&= \begin{pmatrix}
0 & 0 & 0 & 1 \\
0 & 0 & 1 & 0 \\
0 & 1 & 0 & 0 \\
1 & 0 & 0 & 0 
\end{pmatrix}.
\end{align*}
\begin{align*}
Z \otimes Z &= \begin{pmatrix}
1 & 0 \\
0 & -1
\end{pmatrix}^{\otimes 2} \\
&= 
\begin{pmatrix}
1 & 0   & 0  & 0 \\
0 & -1 & 0  & 0 \\
0 & 0   & -1 & 0 \\
0 & 0   & 0   & 1
\end{pmatrix}.
\end{align*}
Now
\[
\begin{pmatrix}
0 & 0 & 0 & 1 \\
0 & 0 & 1 & 0 \\
0 & 1 & 0 & 0 \\
1 & 0 & 0 & 0 
\end{pmatrix}
\begin{pmatrix}
1 & 0   & 0  & 0 \\
0 & -1 & 0  & 0 \\
0 & 0   & -1 & 0 \\
0 & 0   & 0   & 1
\end{pmatrix}
=
\begin{pmatrix}
0 & 0 & 0 & 1 \\
0 & 0 & -1 & 0 \\
0 & -1 & 0 & 0 \\
1 & 0 & 0 & 0 
\end{pmatrix}.
\]
Also 
\[
\begin{pmatrix}
1 & 0   & 0  & 0 \\
0 & -1 & 0  & 0 \\
0 & 0   & -1 & 0 \\
0 & 0   & 0   & 1
\end{pmatrix}
\begin{pmatrix}
0 & 0 & 0 & 1 \\
0 & 0 & 1 & 0 \\
0 & 1 & 0 & 0 \\
1 & 0 & 0 & 0 
\end{pmatrix}
=
\begin{pmatrix}
0 & 0 & 0 & 1 \\
0 & 0 & -1 & 0 \\
0 & -1 & 0 & 0 \\
1 & 0 & 0 & 0 
\end{pmatrix},
\]
which proves that the two matrices commute.

\section*{QCE 7.7}
Consider the eigenvectors in Example 7.4. Show that $[H_I, \vec{\sigma_A} \cdot \vec{\sigma_B}] = 0$, and hence show that the eigenvectors of the Hamiltonian are eigenvectors of the $\vec{\sigma_A} \cdot \vec{\sigma_B}$ operator. In particular, show that $\vec{\sigma_A} \cdot \vec{\sigma_B} \ket{\phi_i} = \ket{\phi_i}$ for $i = 1, 2, 3$ and 
$\vec{\sigma_A} \cdot \vec{\sigma_B} \ket{\phi_4} = -3 \ket{\phi_4}$.

From the Example 7.4 we have that
\[
H_I = 
\frac{\mu^2}{r^3}
\begin{pmatrix}
-2 & 0 & 0 & 0 \\
0 & 2 & 2 & 0 \\
0 & 2 & 2 & 0 \\
0 & 0 & 0 & -2
\end{pmatrix},
\;
\vec{\sigma_A} \cdot \vec{\sigma_B} = 
\begin{pmatrix}
1 & 0 & 0 & 0 \\
0 & -1 & 2 & 0 \\
0 & 2 & -1 & 0 \\
0 & 0 & 0 & 1
\end{pmatrix}.
\]
Condition $[H_I, \vec{\sigma_A} \cdot \vec{\sigma_B}] = 0$ means that the two matrices commute and that is the case as a routine calculation may show. Wolframalpha tells me that the eigenvectors of $H_I$ are
\[
\begin{pmatrix}
0 \\
-1 \\
-1 \\
0
\end{pmatrix}, \qquad
\begin{pmatrix}
0 \\
0 \\
0 \\
1
\end{pmatrix}, \qquad
\begin{pmatrix}
1 \\
0 \\
0 \\
0
\end{pmatrix}, \qquad
\begin{pmatrix}
0 \\
-1 \\
1 \\
0
\end{pmatrix},
\]
and that is not in accord with the eigenvectors of $\vec{\sigma_A} \cdot \vec{\sigma_B}$, thank you, book. :(

Now
\begin{align*}
\vec{\sigma_A} \cdot \vec{\sigma_B} \ket{\phi_1} &= 
\begin{pmatrix}
1 & 0 & 0 & 0 \\
0 & -1 & 2 & 0 \\
0 & 2 & -1 & 0 \\
0 & 0 & 0 & 1
\end{pmatrix} 
\frac{1}{\sqrt{2}}
\begin{pmatrix}
0 \\
1 \\
1 \\
0
\end{pmatrix} \\
   &= 
   \frac{1}{\sqrt{2}}
\begin{pmatrix}
0 \\
1 \\
1 \\
0
\end{pmatrix} \\
&= \ket{\phi_1},
\end{align*}
\begin{align*}
\vec{\sigma_A} \cdot \vec{\sigma_B} \ket{\phi_2} &=
\begin{pmatrix}
1 & 0 & 0 & 0 \\
0 & -1 & 2 & 0 \\
0 & 2 & -1 & 0 \\
0 & 0 & 0 & 1
\end{pmatrix} 
\begin{pmatrix}
0 \\
0 \\
0 \\
1
\end{pmatrix} \\
&= 
\begin{pmatrix}
0 \\
0 \\
0 \\
1 
\end{pmatrix} \\
&= \ket{\phi_2},
\end{align*}
\begin{align*}
\vec{\sigma_A} \cdot \vec{\sigma_B} \ket{\phi_3} &= 
\begin{pmatrix}
1 & 0 & 0 & 0 \\
0 & -1 & 2 & 0 \\
0 & 2 & -1 & 0 \\
0 & 0 & 0 & 1
\end{pmatrix} 
\begin{pmatrix}
1 \\
0 \\
0 \\
0
\end{pmatrix} \\
&= 
\begin{pmatrix}
1 \\
0 \\
0 \\
0
\end{pmatrix} \\
&= \ket{\phi_3},
\end{align*}
\begin{align*}
\vec{\sigma_A} \cdot \vec{\sigma_B} \ket{\phi_4} &= 
\begin{pmatrix}
1 & 0 & 0 & 0 \\
0 & -1 & 2 & 0 \\
0 & 2 & -1 & 0 \\
0 & 0 & 0 & 1
\end{pmatrix}
\frac{1}{\sqrt{2}}
\begin{pmatrix}
0 \\
1 \\
-1 \\
0
\end{pmatrix}  \\
&= 
\frac{1}{\sqrt{2}}
\begin{pmatrix}
0 \\
-3 \\
3 \\
0
\end{pmatrix} \\
&= -3 \ket{\phi_4}.
\end{align*}

\section*{QCE 7,8}
Is the state $X \otimes Z \ket{\beta_{00}}$ entangled?
\begin{align*}
X \otimes Z \ket{00} &= 
\begin{pmatrix}
0 & 1 \\
1 & 0 
\end{pmatrix}
\otimes
\begin{pmatrix}
1 & 0 \\
0 & - 1
\end{pmatrix}
\frac{1}{\sqrt{2}}
\begin{pmatrix}
1 \\ 0 \\ 0 \\ 1
\end{pmatrix} \\
&= 
\frac{1}{\sqrt{2}}
\begin{pmatrix}
0 & 0   & 1 & 0 \\
0 & 0   & 0 & -1 \\
1 & 0   & 0 & 0 \\
0 & -1 & 0 & 0 
\end{pmatrix}
\begin{pmatrix}
1 \\ 0 \\ 0 \\ 1
\end{pmatrix} \\
&= \frac{1}{\sqrt{2}}
\begin{pmatrix}
0 \\ -1 \\ 1 \\ 0
\end{pmatrix} \\
&= \frac{-\ket{01} + \ket{10}}{\sqrt{2}} \\
&= -\ket{\beta_{11}},
\end{align*}
so the state is entangled but is not a Bell state.

\section*{QCE 7.9}
Find the Pauli representation of 
\[
\rho = \begin{pmatrix}
\sin^2 \theta                             & e^{-i\phi} \sin \theta \cos \theta \\
e^{i\phi} \sin \theta \cos \theta & \cos^2 \theta
\end{pmatrix}.
\]
First we need to compute 
\begin{align*}
c_0 &= Tr(\rho \sigma_0) \\
      &= Tr(\rho) \\
      &= \sin^2 \theta + \cos^2 \theta \\
      &= 1,
\end{align*}
\begin{align*}
c_1 &= Tr(\rho \sigma_1) \\
      &= Tr(\rho \sigma_x) \\
      &= 
      Tr \begin{pmatrix}
          \sin^2 \theta                             & e^{-i\phi} \sin \theta \cos \theta \\
          e^{i\phi} \sin \theta \cos \theta & \cos^2 \theta
      \end{pmatrix}
      \begin{pmatrix}
      0 & 1 \\
      1 & 0
      \end{pmatrix} \\
      &= Tr
      \begin{pmatrix}
      e^{-i\phi} \sin \theta \cos \theta & \sin^2 \theta \\
      \cos^2 \theta & e^{i\phi} \sin \theta \cos \theta
      \end{pmatrix} \\
      &= 
      e^{-i\phi} \sin \theta \cos \theta + e^{i\phi} \sin \theta \cos \theta \\ 
      &=
      (\sin \theta \cos \theta)(\cos \phi -i \sin \phi + \cos \phi + i \sin \phi) \\
      &= 2 \sin \theta \cos \theta \cos \phi,
\end{align*}
\begin{align*}
c_2 &= Tr(\rho \sigma_2) \\
      &= Tr(\rho \sigma_y) \\
      &= 
      Tr \begin{pmatrix}
          \sin^2 \theta                             & e^{-i\phi} \sin \theta \cos \theta \\
          e^{i\phi} \sin \theta \cos \theta & \cos^2 \theta
      \end{pmatrix}
      \begin{pmatrix}
      0 & -i \\
      i & 0
      \end{pmatrix} \\
      &=  Tr 
      \begin{pmatrix}
      ie^{-i\phi} \sin \theta \cos \theta & -i\sin^2 \theta \\
      i \cos^2 \theta & -ie^{i\phi} \sin \theta \cos \theta
      \end{pmatrix} \\
      &= ie^{-i\phi} \sin \theta \cos \theta - ie^{i\phi} \sin \theta \cos \theta \\
      &= \sin \theta \cos \theta (ie^{-i\phi} - ie^{i\phi}) \\
	  &= \sin \theta \cos \theta (i(\cos \phi - i \sin \phi) -i (\cos \phi + i \sin \phi)) \\
	  &= \sin \theta \cos \theta ( i\cos \phi -i^2 \sin \phi - i\cos \phi - i^2 \sin \phi ) \\
	  &= 2 \sin \theta \cos \theta \sin \phi.
\end{align*}
\begin{align*}
c_3 &= Tr(\rho \sigma_3) \\
      &= Tr(\rho \sigma_z) \\
      &= Tr
      \begin{pmatrix}
          \sin^2 \theta                             & e^{-i\phi} \sin \theta \cos \theta \\
          e^{i\phi} \sin \theta \cos \theta & \cos^2 \theta
      \end{pmatrix}
      \begin{pmatrix}
      1 & 0 \\
      0 & -1 
      \end{pmatrix} \\
      &= Tr
      \begin{pmatrix}
      \sin^2 \theta & -e^{-i\phi} \sin \theta \cos \theta \\
      e^{i\phi} \sin \theta \cos \theta & -\cos^2 \theta
      \end{pmatrix} \\
      &= \sin^2 \theta - \cos^2 \theta.
\end{align*}
The Pauli representation for the system is
\[
\rho = \sum_{i = 0}^{3} Tr(\sigma_i) \sigma_i.
\]

\section*{QCE 7.10}
Use (7.36) to show that $\ket{\beta_{10}}$ is entangled. Apply the same criterion to test $X \otimes Z\ket{\beta_{00}}$.

Here we have
\[
\ket{\beta_{00}} = \frac{\ket{00} + \ket{11}}{\sqrt{2}}, \qquad
\ket{\beta_{10}} = \frac{\ket{00} - \ket{11}}{\sqrt{2}}.
\]
The density operator for $\ket{\beta_{10}}$ is
\[
\rho = \frac{1}{2}(\ket{00} - \ket{11})(\bra{00} - \bra{11}) = \frac{1}{2} (\ket{00} \bra{00} - \ket{00} \bra{11} - \ket{11} \bra{00} + \ket{11} \bra{11}),
\]
and its density matrix is
\[
\frac{1}{2} \begin{pmatrix}
1  & 0  & 0 & -1 \\
0  & 0  & 0 & 0 \\
0  & 0  & 0 & 0 \\
-1 & 0 & 0 & 1
\end{pmatrix}.
\]
Next we find
\begin{align*}
c_{11} &= Tr(\rho X \otimes X) = Tr \frac{1}{2} 
\begin{pmatrix}
1 & 0 & 0 & -1 \\
0 & 0 & 0 & 0 \\
0 & 0 & 0 & 0 \\
-1 & 0 & 0 & 1 
\end{pmatrix}
\begin{pmatrix}
0 & 0 & 0 & 1 \\
0 & 0 & 1 & 0 \\
0 & 1 & 0 & 0 \\
1 & 0 & 0 & 0 
\end{pmatrix} \\
&=
\frac{1}{2} Tr
\begin{pmatrix}
-1 & 0 & 0 & 1 \\
0 & 0 & 0 & 0 \\
0 & 0 & 0 & 0 \\
1 & 0 & 0 & -1 
\end{pmatrix} \\
&= -1
\end{align*}
\begin{align*}
c_{22} &= Tr(\rho Y \otimes Y) = Tr \frac{1}{2} 
\begin{pmatrix}
1 & 0 & 0 & -1 \\
0 & 0 & 0 & 0 \\
0 & 0 & 0 & 0 \\
-1 & 0 & 0 & 1 
\end{pmatrix}
\begin{pmatrix}
0 & 0 & 0 & -1 \\
0 & 0 & 1 & 0 \\
0 & 1 & 0 & 0 \\
-1 & 0 & 0 & 0 
\end{pmatrix} \\
&=
\frac{1}{2} Tr
\begin{pmatrix}
1 & 0 & 0 & -1 \\
0 & 0 & 0 & 0 \\
0 & 0 & 0 & 0 \\
-1 & 0 & 0 & 1 
\end{pmatrix} \\
&= 1
\end{align*}
\begin{align*}
c_{33} &= Tr(\rho Z \otimes Z) = 
Tr \frac{1}{2} 
\begin{pmatrix}
1 & 0 & 0 & -1 \\
0 & 0 & 0 & 0 \\
0 & 0 & 0 & 0 \\
-1 & 0 & 0 & 1 
\end{pmatrix}
\begin{pmatrix}
1 & 0 & 0 & 0 \\
0 & -1 & 0 & 0 \\
0 & 0 & -1 & 0 \\
0 & 0 & 0 & 1 
\end{pmatrix} \\
&=
\frac{1}{2} Tr
\begin{pmatrix}
1 & 0 & 0 & -1 \\
0 & 0 & 0 & 0 \\
0 & 0 & 0 & 0 \\
-1 & 0 & 0 & 1 
\end{pmatrix} \\
&= 1.
\end{align*}
Now we see that $|c_{11}| + |c_{22}| + |c_{33}| = 3$, so the state is entangled.

\begin{align*}
X \otimes Z \ket{\beta_{00}} &= 
\begin{pmatrix}
0 & 1 \\
1 & 0
\end{pmatrix}
\otimes
\begin{pmatrix}
1 & 0 \\
0 & -1 
\end{pmatrix}
\frac{1}{\sqrt{2}}
\begin{pmatrix}
1 \\ 0 \\ 0 \\ 1
\end{pmatrix} \\
&= 
\frac{1}{\sqrt{2}}
\begin{pmatrix}
0 & 0 & 1 & 0 \\
0 & 0 & 0 & -1 \\
1 & 0 & 0 & 0 \\
0 & -1 & 0 & 0
\end{pmatrix}
\begin{pmatrix}
1 \\ 0 \\ 0 \\ 1
\end{pmatrix} \\
&= 
\frac{1}{\sqrt{2}}
\begin{pmatrix}
0 \\ -1 \\ 1 \\ 0
\end{pmatrix} \\
&= \frac{-\ket{01} + \ket{10}}{\sqrt{2}}.
\end{align*}
Now the density operator/matrix for the above state is 
\begin{align*}
\rho &= \frac{1}{2} (-\ket{01} + \ket{10}) (-\bra{01} + \bra{10}) \\
        &= \frac{1}{2} ( \ket{01} \bra{01} - \ket{01} \bra{10} - \ket{10} \bra{01} + \ket{10} \bra{10} ) \\
        &= 
	    \frac{1}{2}        
        \begin{pmatrix}
        0 & 0   & 0   & 0 \\
        0 & 1   & -1 & 0 \\
        0 & -1 & 1   & 0 \\
        0 & 0   & 0   & 0 
        \end{pmatrix}.
\end{align*}
Next we find
\begin{align*}
c_{11} &= Tr (\rho X \otimes X) = \frac{1}{2} Tr 
\begin{pmatrix}
        0 & 0   & 0   & 0 \\
        0 & 1   & -1 & 0 \\
        0 & -1 & 1   & 0 \\
        0 & 0   & 0   & 0 
\end{pmatrix}
\begin{pmatrix}
0 & 0 & 0 & 1 \\
0 & 0 & 1 & 0 \\
0 & 1 & 0 & 0 \\
1 & 0 & 0 & 0 
\end{pmatrix} \\
&= \frac{1}{2} Tr 
\begin{pmatrix}
0 & 0 & 0 & 0 \\
0 & -1 & 1 & 0 \\
0 & 1 & -1 & 0 \\
0 & 0 & 0 & 0 
\end{pmatrix} = -1,
\end{align*}
\begin{align*}
c_{22} &= Tr (\rho Y \otimes Y) = \frac{1}{2} Tr 
\begin{pmatrix}
        0 & 0   & 0   & 0 \\
        0 & 1   & -1 & 0 \\
        0 & -1 & 1   & 0 \\
        0 & 0   & 0   & 0 
\end{pmatrix}
\begin{pmatrix}
0 & 0 & 0 & -1 \\
0 & 0 & 1 & 0 \\
0 & 1 & 0 & 0 \\
-1 & 0 & 0 & 0 
\end{pmatrix} \\
&= \frac{1}{2} Tr 
\begin{pmatrix}
0 & 0 & 0 & 0 \\
0 & -1 & 1 & 0 \\
0 & 1 & -1 & 0 \\
0 & 0 & 0 & 0 
\end{pmatrix} = -1,
\end{align*}
\begin{align*}
c_{33} &= Tr (\rho Z \otimes Z) = \frac{1}{2} Tr 
\begin{pmatrix}
        0 & 0   & 0   & 0 \\
        0 & 1   & -1 & 0 \\
        0 & -1 & 1   & 0 \\
        0 & 0   & 0   & 0 
\end{pmatrix}
\begin{pmatrix}
1 & 0 & 0 & 0 \\
0 & -1 & 0 & 0 \\
0 & 0 & -1 & 0 \\
0 & 0 & 0 & 1 
\end{pmatrix} \\
&= \frac{1}{2} Tr 
\begin{pmatrix}
0 & 0 & 0 & 0 \\
0 & -1 & 1 & 0 \\
0 & 1 & -1 & 0 \\
0 & 0 & 0 & 0 
\end{pmatrix} = -1.
\end{align*}
Once again we have $|c_{11}| + |c_{22}| + |c_{33}| = 3$ so this state is entangled as well.

\section*{QCE 7.11}
Derive 
\[
\ket{\beta_{00}} \bra{\beta_{00}} = \frac{1}{4} (I \otimes I + X \otimes X - Y \otimes Y + Z \otimes Z).
\]
We have that 
\[
\ket{\beta_{00}} = \frac{\ket{00} + \ket{11}}{\sqrt{2}},
\]
so
\begin{align*}
\ket{\beta_{00}} \bra{\beta_{00}} &= \frac{1}{2} (\ket{00} + \ket{11})(\bra{00} + \bra{11}) \\
		&= \frac{1}{2} (\ket{00} \bra{00} + \ket{00} \bra{11} + \ket{11} \bra{00} + \ket{11} \bra{11}) \\
		&= \frac{1}{2}
		\begin{pmatrix}
		1 & 0 & 0 & 1 \\
		0 & 0 & 0 & 0 \\
		0 & 0 & 0 & 0 \\
		1 & 0 & 0 & 1 
		\end{pmatrix}.
\end{align*}
Now
\begin{align*}
I \otimes I + X \otimes X - Y \otimes Y + Z \otimes Z &=
\begin{pmatrix}
1 & 0 & 0 & 0 \\
0 & 1 & 0 & 0 \\
0 & 0 & 1 & 0 \\
0 & 0 & 0 & 1 
\end{pmatrix} 
+
\begin{pmatrix}
0 & 0 & 0 & 1 \\
0 & 0 & 1 & 0 \\
0 & 1 & 0 & 0 \\
1 & 0 & 0 & 0 
\end{pmatrix} \\
&- 
\begin{pmatrix}
0 & 0 & 0 & - 1 \\
0 & 0 & 1 & 0 \\
0 & 1 & 0 & 0 \\
-1 & 0 & 0 & 0 
\end{pmatrix}
+ 
\begin{pmatrix}
1 & 0 & 0 & 0 \\
0 & -1 & 0 & 0 \\
0 & 0 & -1 & 0 \\
0 & 0 & 0 & 1
\end{pmatrix} \\
&= 
\begin{pmatrix}
1 & 0 & 0 & 1 \\
0 & 1 & 1 & 0 \\
0 & 1 & 1 & 0 \\
1 & 0 & 0 & 1 
\end{pmatrix}
+
\begin{pmatrix}
1 & 0 & 0 & 1 \\
0 & -1 & -1 & 0 \\
0 & -1 & -1 & 0 \\
1 & 0 & 0 & 1 
\end{pmatrix} \\
&= 
\begin{pmatrix}
2 & 0 & 0 & 2 \\
0 & 0 & 0 & 0 \\
0 & 0 & 0 & 0 \\
2 & 0 & 0 & 2 
\end{pmatrix} \\
&= 4\ket{\beta_{00}} \bra{\beta_{00}},
\end{align*}
which yields the desired result.

\section*{QCE 7.12}
Can the following state be written in diagonal form in terms of the Bell basis?
\[
\rho = \begin{pmatrix}
\frac{1}{2} & 0 & 0 & -\frac{1}{8} \\
0 & 0 & 0 & 0 \\
0 & 0 & 0 & 0 \\
-\frac{1}{8} & 0 & 0 & \frac{1}{2} 
\end{pmatrix}
\]
Using (7.43), determine if this state is a separable state. 

Now
\begin{align*}
\rho &= \frac{1}{2} \Bigg( \ket{00}\bra{00} + \ket{11} \bra{11} \Bigg) - \frac{1}{8} \Bigg( \ket{00} \bra{11} + \ket{11} \bra{00} \Bigg) \\
        &= \frac{1}{2} \Bigg( \ket{\beta_{00}} \bra{\beta_{00}} + \ket{\beta_{10}} \bra{\beta_{10}} \Bigg) - \frac{1}{8} \Bigg( \ket{\beta_{00}} \bra{\beta_{00}} - \ket{\beta_{10}} \bra{\beta_{10}} \Bigg) \\
        &= \frac{3}{8} \Bigg( \ket{\beta_{00}} \bra{\beta_{00}} \Bigg) + \frac{5}{8} \Bigg( \ket{\beta_{10}} \bra{\beta_{10}} \Bigg).
\end{align*}
Since $c_{00} = \frac{3}{8} \leq \frac{1}{2}$, this state is separable.

\section*{QCE 7.13}
Verify that 
\[
\ket{\psi} = \Bigg( \frac{\ket{0} - \ket{1}}{\sqrt{2}}\Bigg) \otimes \Bigg( \frac{\ket{0} - \ket{1}}{\sqrt{2}}\Bigg) 
\]
is a product state using (7.36).

A product state is also called separable. Now 
\begin{align*}
\ket{\psi} &= \Bigg( \frac{\ket{0} - \ket{1}}{\sqrt{2}}\Bigg) \otimes \Bigg( \frac{\ket{0} - \ket{1}}{\sqrt{2}}\Bigg) \\
                &= \frac{1}{2} (\ket{00} - \ket{01} - \ket{10} + \ket{11}).
\end{align*}
Now
\begin{align*}
\rho &= \ket{\psi} \bra{\psi} \\
        &= \frac{1}{4}  (\ket{00} - \ket{01} - \ket{10} + \ket{11})  (\bra{00} - \bra{01} - \bra{10} + \bra{11}) \\
        &= \frac{1}{4} \Bigg( \ket{00} \bra{00} - \ket{00} \bra{01} -\ket{00} \bra{10} + \ket{00} \bra{11} \\
        &- \ket{01} \bra{00} + \ket{01} \bra{01} + \ket{01} \bra{10} - \ket{01} \bra{11} \\
        &- \ket{10} \bra{00} + \ket{10} \bra{01} + \ket{10} \bra{10} - \ket{10} \bra{11} \\
        &+ \ket{11} \bra{00} - \ket{11} \bra{01} - \ket{11} \bra{10} + \ket{11} \bra{11} \Bigg) \\
        &=
        \frac{1}{4} 
        \begin{pmatrix}
        1 & -1 & -1 & 1 \\
        -1 & 1 & 1 & -1 \\
        -1 & 1 & 1 & -1 \\
        1 & -1 & -1 & 1
        \end{pmatrix}.
\end{align*}
The first term is
\begin{align*}
c_{11} &= Tr(\rho X \otimes X) \\
          &= \frac{1}{4} Tr
        \begin{pmatrix}
        1 & -1 & -1 & 1 \\
        -1 & 1 & 1 & -1 \\
        -1 & 1 & 1 & -1 \\
        1 & -1 & -1 & 1
        \end{pmatrix}
        \begin{pmatrix}
        0 & 0 & 0 & 1 \\
        0 & 0 & 1 & 0 \\
        0 & 1 & 0 & 0 \\
        1 & 0 & 0 & 0 \\
        \end{pmatrix} \\
        &= 
        \frac{1}{4} Tr
        \begin{pmatrix}
        1 & -1 & -1 & 1 \\
        -1 & 1 & 1 & -1 \\
        -1 & 1 & 1 & -1 \\
        1 & -1 & -1 & 1
        \end{pmatrix} \\
        &= 1.
\end{align*}
\begin{align*}
c_{22} &= Tr(\rho Y \otimes Y) \\
          &= \frac{1}{4} Tr
        \begin{pmatrix}
        1 & -1 & -1 & 1 \\
        -1 & 1 & 1 & -1 \\
        -1 & 1 & 1 & -1 \\
        1 & -1 & -1 & 1
        \end{pmatrix}
        \begin{pmatrix}
        0   & 0 & 0 & -1 \\
        0   & 0 & 1 & 0 \\
        0   & 1 & 0 & 0 \\
        -1 & 0 & 0 & 0 \\
        \end{pmatrix} \\
        &= \frac{1}{4} Tr
        \begin{pmatrix}
        -1 & -1 & -1 & -1 \\
        1 & 1 & 1 & 1 \\
        1 & 1 & 1 & 1 \\
        -1 & -1 & -1 & -1
        \end{pmatrix} \\
        &= 0,
\end{align*}
\begin{align*}
c_{33} &= Tr(\rho Z \otimes Z) \\
          &= \frac{1}{4} Tr
        \begin{pmatrix}
        1 & -1 & -1 & 1 \\
        -1 & 1 & 1 & -1 \\
        -1 & 1 & 1 & -1 \\
        1 & -1 & -1 & 1
        \end{pmatrix}
        \begin{pmatrix}
        1 & 0   & 0   & 0 \\
        0 & -1 & 0   & 0 \\
        0 & 0   & -1 & 0 \\
        0 & 0   & 0   & 1 \\
        \end{pmatrix} \\
        &= \frac{1}{4} Tr
        \begin{pmatrix}
        1 & 1 & 1 & 1 \\
        -1 & -1 & -1 & -1 \\
        -1 & -1 & -1 & -1 \\
        1 & 1 & 1 & 1 
        \end{pmatrix} \\
        &= 0.
\end{align*}
As $|c_{11}| + |c_{22}| + |c_{33}| = 1 \leq 1$, it is a product state in question.

\section*{QCE 7.14}
Verify that the state
\[
\ket{\psi} = \frac{1}{\sqrt{2}} \ket{\beta_{00}} - \frac{1}{\sqrt{2}} \ket{\beta_{01}}
\]
is entangled by calculating the Schmidt number.

First, let us rewrite the state:
\begin{align*}
\ket{\psi} &= \frac{1}{\sqrt{2}} \ket{\beta_{00}} - \frac{1}{\sqrt{2}} \ket{\beta_{01}} \\
                &= \frac{1}{2} (\ket{00} + \ket{11}) - \frac{1}{2} (\ket{01} + \ket{10}) \\
                &= \frac{1}{2} (\ket{00} - \ket{01} - \ket{10} + \ket{11}).
\end{align*}
Now
\begin{align*}
\rho &= \ket{\psi} \bra{\psi} \\
        &= \frac{1}{4} \Bigg( \ket{00} - \ket{01} - \ket{10} + \ket{11} \Bigg) \Bigg( \bra{00} - \bra{01} - \bra{10} + \bra{11} \Bigg) \\
        &= \frac{1}{4} \Bigg( \ket{00} \bra{00} -  \ket{00} \bra{01} - \ket{00} \bra{10} +  \ket{00} \bra{11} \\
        &-  \ket{01} \bra{00} +  \ket{01} \bra{01} +  \ket{01} \bra{10} - \ket{01} \bra{11} \\ 
        &-  \ket{10} \bra{00} +  \ket{10} \bra{01} +  \ket{10} \bra{10} -  \ket{10} \bra{11} \\
        &+  \ket{11} \bra{00} -  \ket{11} \bra{01} - \ket{11} \bra{10} +  \ket{11} \bra{11} \Bigg) \\
        &= 
        \frac{1}{4}
        \begin{pmatrix}
        1 & -1 & -1 & 1 \\
        -1 & 1 & 1 & -1 \\
        -1 & 1 & 1 & -1 \\
        1 & -1 & -1 & 1
        \end{pmatrix}.
\end{align*}
Now
\begin{align*}
\rho' &= Tr(\ket{\psi} \bra{\psi}) \\
         &= \braket{0|\psi} \braket{\psi|0} + \braket{1|\psi} \braket{\psi|1} \\
         &= \frac{1}{4} ( \ket{0} \bra{0} - \ket{0} \bra{1} - \ket{1} \bra{0} +\ket{1} \bra{1} ) \\
         &+ \frac{1}{4} ( \ket{0} \bra{0} - \ket{0} \bra{1} - \ket{1} \bra{0} + \ket{1} \bra{1} ) \\
         &= \frac{1}{2} (\ket{0} \bra{0} - \ket{0} \bra{1} - \ket{1} \bra{0} + \ket{1} \bra{1}) \\
         &= 
         \frac{1}{2}
         \begin{pmatrix}
         1 & - 1 \\
         -1 & 1 
         \end{pmatrix}.
\end{align*}
On page 169 of the book, it is told that the above matrix has eigenvalues $\lambda_1 = 1, \lambda_2 = 0$, so the Schmidt number is one and this is a separable state.

\section*{QCE 8.1}
Describe the action of the Y gate in terms of the Bloch sphere picture.

We have 
\[
\ket{\psi} = \cos \theta \ket{0} + e^{i\phi} \sin \theta \ket{1} = 
\begin{pmatrix}
\cos \theta \\
e^{i\phi} \sin \theta
\end{pmatrix},
 \qquad Y =
\begin{pmatrix}
0 & -i \\
i  & 0
\end{pmatrix}.
\]
Now
\begin{align*}
Y\ket{\psi} &= 
\begin{pmatrix}
0 & -i \\
i  & 0
\end{pmatrix}
\begin{pmatrix}
\cos \theta \\
e^{i\phi} \sin \theta
\end{pmatrix} \\
&= 
\begin{pmatrix}
-ie^{i\phi} \sin \theta \\
i \cos \theta
\end{pmatrix} \\
&= 
\begin{pmatrix}
-i ( \cos \phi + i \sin \phi ) \sin \theta \\
i \cos \theta
\end{pmatrix} \\
&=
\begin{pmatrix}
-i \cos \phi \sin \theta + \sin \phi \sin \theta \\
i \cos \theta
\end{pmatrix}
\end{align*}

\section*{QCE 8.2}
\[
X^{11} =
\begin{pmatrix}
1 & 0 \\
0 & 0
\end{pmatrix},
\qquad
X^{12} =
\begin{pmatrix}
0 & 1 \\
0 & 0
\end{pmatrix},
\qquad
X^{21} =
\begin{pmatrix}
0 & 0 \\
1 & 0
\end{pmatrix},
\qquad
X^{22} =
\begin{pmatrix}
0 & 0 \\
0 & 1
\end{pmatrix}.
\]
The Hadamard basis is
\[
\ket{+} = 
\frac{1}{\sqrt{2}}
\begin{pmatrix}
1 \\
1
\end{pmatrix}, \qquad
\ket{-} = 
\frac{1}{\sqrt{2}}
\begin{pmatrix}
1 \\
-1
\end{pmatrix}, 
\]
so
\[
X^{11} \ket{+} =
\frac{1}{\sqrt{2}} \ket{0}, \;
X^{12} \ket{+} =
\frac{1}{\sqrt{2}} \ket{0}, \;
X^{21} \ket{+} =
\frac{1}{\sqrt{2}} \ket{1}, \;
X^{22} \ket{+} =
\frac{1}{\sqrt{2}} \ket{1},
\]
and
\[
X^{11} \ket{-} =
\frac{1}{\sqrt{2}} \ket{0}, \;
X^{12} \ket{-} =
- \frac{1}{\sqrt{2}} \ket{0}, \;
X^{21} \ket{-} =
\frac{1}{\sqrt{2}} \ket{1}, \;
X^{22} \ket{-} =
-\frac{1}{\sqrt{2}} \ket{1}.
\]

\section*{QCE 8.3}
Find a way to write the Pauli operators X, Y and Z in terms of the Hubbard operators.

This is a basic matrix shit:
\begin{align*}
X &=
\begin{pmatrix}
0 & 1 \\
1 & 0
\end{pmatrix}
= X^{12} + X^{21}, \\
Y &= \begin{pmatrix}
0 & -i \\
i  & 0 
\end{pmatrix}
= -i X^{12} + i X^{21} = i(X^{21} - X^{12}), \\
Z &= \begin{pmatrix}
1 & 0 \\
0 & -1
\end{pmatrix} 
= X^{11} - X^{22}.
\end{align*}

\section*{QCE 8.4}
Show that the controlled NOT gate is Hermitian and unitary.
\[
\begin{pmatrix}
1 & 0 & 0 & 0 \\
0 & 1 & 0 & 0 \\
0 & 0 & 0 & 1 \\
0 & 0 & 1 & 0
\end{pmatrix}
= 
\begin{pmatrix}
1 & 0 & 0 & 0 \\
0 & 1 & 0 & 0 \\
0 & 0 & 0 & 1 \\
0 & 0 & 1 & 0
\end{pmatrix}^{\dag},
\]
so the matrix is Hermitian. Also
\begin{align*}
\begin{pmatrix}
1 & 0 & 0 & 0 \\
0 & 1 & 0 & 0 \\
0 & 0 & 0 & 1 \\
0 & 0 & 1 & 0
\end{pmatrix}
\begin{pmatrix}
1 & 0 & 0 & 0 \\
0 & 1 & 0 & 0 \\
0 & 0 & 0 & 1 \\
0 & 0 & 1 & 0
\end{pmatrix}
= 
\begin{pmatrix}
1 & 0 & 0 & 0 \\
0 & 1 & 0 & 0 \\
0 & 0 & 1 & 0 \\
0 & 0 & 0 & 1
\end{pmatrix},
\end{align*}
so the matrix is unitary.

\section*{QCE 8.5}
Let $\ket{a} = \ket{1}$, and consider the circuit shown in Figure 8.5. Determine which Bell states are generated as output when $\ket{b} = \ket{0}$, $\ket{b} = \ket{1}$.

The Hadamard gate converts $\ket{a} = \ket{1}$ to $\ket{-} = \frac{1}{\sqrt{2}} (\ket{0} - \ket{1})$. Also the matrix representation of CNOT is
\[
\begin{pmatrix}
1 & 0 & 0 & 0 \\
0 & 1 & 0 & 0 \\
0 & 0 & 0 & 1 \\
0 & 0 & 1 & 0
\end{pmatrix}.
\]
For the first case, we have $\ket{b} = \ket{0}$, so the input qubit is 
\[
\ket{a} \ket{b} = \frac{1}{\sqrt{2}} (\ket{0} - \ket{1})\ket{0} = \frac{1}{\sqrt{2}} (\ket{00} - \ket{10} = 
\frac{1}{\sqrt{2}}
\begin{pmatrix}
1 \\
0 \\
-1 \\
0
\end{pmatrix},
\]
and
\begin{align*}
CNOT \ket{ab} &= 
\frac{1}{\sqrt{2}}
\begin{pmatrix}
1 & 0 & 0 & 0 \\
0 & 1 & 0 & 0 \\
0 & 0 & 0 & 1 \\
0 & 0 & 1 & 0
\end{pmatrix}
\begin{pmatrix}
1 \\
0 \\
-1 \\
0
\end{pmatrix} \\
&= \frac{1}{\sqrt{2}}
\begin{pmatrix}
1 \\
0 \\
0 \\
-1
\end{pmatrix} \\
&= \ket{\beta_{10}}.
\end{align*}
For the second case, we have $\ket{b} = \ket{1}$, so the input qubit is 
\[
\ket{a} \ket{b} = \frac{1}{\sqrt{2}} (\ket{0} - \ket{1})\ket{1} = \frac{1}{\sqrt{2}} (\ket{01} - \ket{11} = 
\frac{1}{\sqrt{2}}
\begin{pmatrix}
0 \\
1 \\
0 \\
-1
\end{pmatrix},
\]
and
\begin{align*}
CNOT \ket{ab} &= 
\frac{1}{\sqrt{2}}
\begin{pmatrix}
1 & 0 & 0 & 0 \\
0 & 1 & 0 & 0 \\
0 & 0 & 0 & 1 \\
0 & 0 & 1 & 0
\end{pmatrix}
\begin{pmatrix}
0 \\
1 \\
0 \\
-1
\end{pmatrix} \\
&= \frac{1}{\sqrt{2}}
\begin{pmatrix}
0 \\
1 \\
-1 \\
0
\end{pmatrix} \\
&= \ket{\beta_{11}}.
\end{align*}

\section*{QCE 8.6}
Write down the matrix representation for the controlled Z gate. Then write down its representation using Dirac notation.

The matrix representation is
\[
\begin{pmatrix}
1 & 0 & 0 & 0 \\
0 & 1 & 0 & 0 \\
0 & 0 & 1 & 0 \\
0 & 0 & 0 & -1 \\
\end{pmatrix},
\]
and in Dirac notation this is $\ket{00} \bra{00} + \ket{01} \bra{01}  + \ket{10} \bra{10} - \ket{11} \bra{11}$.

\section*{QCE 8.7}
\begin{align*}
X^2 &= 
\begin{pmatrix}
0 & 1 \\
1 & 0 
\end{pmatrix}
\begin{pmatrix}
0 & 1 \\
1 & 0 
\end{pmatrix} \\
&=
\begin{pmatrix}
1 & 0 \\
0 & 1
\end{pmatrix} \\
&= I,
\end{align*}
\begin{align*}
Y^2 &= 
\begin{pmatrix}
0 & -i \\
i  & 0
\end{pmatrix}
\begin{pmatrix}
0 & -i \\
i  & 0
\end{pmatrix} \\
&=
\begin{pmatrix}
1 & 0 \\
0 & 1
\end{pmatrix} \\
&= I,
\end{align*}
\begin{align*}
Z^2 &= 
\begin{pmatrix}
1 &  0 \\
0  & -1
\end{pmatrix}
\begin{pmatrix}
1 & 0 \\
0 & -1
\end{pmatrix} \\
&=
\begin{pmatrix}
1 & 0 \\
0 & 1
\end{pmatrix} \\
&= I,
\end{align*}
\begin{align*}
S^2 &= 
\begin{pmatrix}
1 &  0 \\
0  & i
\end{pmatrix}
\begin{pmatrix}
1 & 0 \\
0 & i
\end{pmatrix} \\
&=
\begin{pmatrix}
1 & 0 \\
0 & -1
\end{pmatrix} \\
&= Z,
\end{align*}
\begin{align*}
T^2 &= 
\begin{pmatrix}
1 &  0 \\
0  & e^{i\pi / 4}
\end{pmatrix}
\begin{pmatrix}
1 &  0 \\
0  & e^{i\pi / 4}
\end{pmatrix} \\
&=
\begin{pmatrix}
1 & 0 \\
0 & e^{i\pi / 4} e^{i\pi / 4}
\end{pmatrix} \\
&=
\begin{pmatrix}
1 & 0 \\
0 & e^{2i\pi / 4}
\end{pmatrix} \\
&=
\begin{pmatrix}
1 & 0 \\
0 & e^{i\pi / 2}
\end{pmatrix} \\
&=
\begin{pmatrix}
1 & 0 \\
0 & \cos \frac{\pi}{2} + i \sin \frac{\pi}{2}
\end{pmatrix} \\
&=
\begin{pmatrix}
1 & 0 \\
0 & i
\end{pmatrix} \\
&= S.
\end{align*}

\section*{QCE 8.10}
By using the tensor product methods developed in chapter 4, show that the controlled-NOT matrix can be generated from $P_0 \otimes I + P_1 \otimes X$.

\begin{align*}
P_0 \otimes I + P_1 \otimes X &= 
\begin{pmatrix}
1 & 0 \\
0 & 0
\end{pmatrix}
\otimes 
\begin{pmatrix}
1 & 0 \\
0 & 1
\end{pmatrix}
+ 
\begin{pmatrix}
0 & 0 \\
0 & 1
\end{pmatrix}
\otimes
\begin{pmatrix}
0 & 1 \\
1 & 0
\end{pmatrix} \\
&= 
\begin{pmatrix}
1 & 0 & 0 & 0 \\
0 & 1 & 0 & 0 \\
0 & 0 & 0 & 0 \\
0 & 0 & 0 & 0 
\end{pmatrix}
+
\begin{pmatrix}
0 & 0 & 0 & 0 \\
0 & 0 & 0 & 0 \\
0 & 0 & 0 & 1 \\
0 & 0 & 1 & 0 
\end{pmatrix} \\
&= 
\begin{pmatrix}
1 & 0 & 0 & 0 \\
0 & 1 & 0 & 0 \\
0 & 0 & 0 & 1 \\
0 & 0 & 1 & 0 
\end{pmatrix} \\
&= CNOT.
\end{align*}

\section*{QCE 8.11}
The operator matrix of the left circuit is that of CNOT:
\[
\begin{pmatrix}
1 & 0 & 0 & 0 \\
0 & 1 & 0 & 0 \\
0 & 0 & 0 & 1 \\
0 & 0 & 1 & 0
\end{pmatrix}.
\]
The two ``parallel'' Hadamard gates may be represented as
\[
H^{\otimes 2} = 
\frac{1}{\sqrt{2}}
\begin{pmatrix}
1 & 1   & 1   & 1 \\
1 & -1 & 1   & -1 \\
1 & 1   & -1 & -1 \\
1 & -1 & -1 & 1 \\
\end{pmatrix},
\]
so
\begin{align*}
H^{\otimes 2} \times CNOT \times H^{\otimes 2} &= 
\frac{1}{4}
\begin{pmatrix}
1 & 1   & 1   & 1 \\
1 & -1 & 1   & -1 \\
1 & 1   & -1 & -1 \\
1 & -1 & -1 & 1 \\
\end{pmatrix}
\begin{pmatrix}
1 & 0 & 0 & 0 \\
0 & 1 & 0 & 0 \\
0 & 0 & 0 & 1 \\
0 & 0 & 1 & 0
\end{pmatrix}
\begin{pmatrix}
1 & 1   & 1   & 1 \\
1 & -1 & 1   & -1 \\
1 & 1   & -1 & -1 \\
1 & -1 & -1 & 1 \\
\end{pmatrix} \\
&= 
\frac{1}{4}
\begin{pmatrix}
4 & 0 & 0 & 0 \\
0 & 4 & 0 & 0 \\
0 & 0 & 0 & 4 \\
0 & 0 & 4 & 0
\end{pmatrix} \\
&= CNOT.
\end{align*}

\section*{QCE 8.12}
\begin{align*}
V &= (1 - i) \frac{I + iX}{2} \\
   &= \frac{(1 - i)}{2} 
\Bigg( 
\begin{pmatrix}
 1 & 0 \\
 0 & 1  
\end{pmatrix}    
+
\begin{pmatrix}
0 & i \\
i & 0
\end{pmatrix}
\Bigg) \\
&= 
\frac{(1-i)}{2} 
\begin{pmatrix}
1 & i \\
i & 1
\end{pmatrix} \\
&= 
\frac{1}{2}
\begin{pmatrix}
1 - i & 1 + i \\
1 + i & 1 - i
\end{pmatrix}
\end{align*}
Now, the conditional $CV$ gate is 
\[
\begin{pmatrix}
1 & 0 & 0 & 0 \\
0 & 1 & 0 & 0 \\
0 & 0 & 1 - i & 1 + i \\
0 & 0 & 1 + i & 1 - i 
\end{pmatrix},
\]
and the conditional $CV^{\dag}$ is
\[
\begin{pmatrix}
1 & 0 & 0 & 0 \\
0 & 1 & 0 & 0 \\
0 & 0 & 1 + i & 1 - i \\
0 & 0 & 1 - i & 1 + i 
\end{pmatrix}.
\]
Now the entire circuit can be represented as
\[
(I \otimes CV)(CNOT \otimes I)(I \otimes CV^{\dag})(CNOT \otimes I)(I \otimes CV).
\]
\begin{align*}
I \otimes CV &= 
\begin{pmatrix}
1 & 0 \\
0 & 1
\end{pmatrix}
\otimes 
\begin{pmatrix}
1 & 0 & 0 & 0 \\
0 & 1 & 0 & 0 \\
0 & 0 & 1 - i & 1 + i \\
0 & 0 & 1 + i & 1 - i 
\end{pmatrix} \\
&= 
\begin{pmatrix}
1 & 0 & 0 & 0 & 0 & 0 & 0 & 0 \\
0 & 1 & 0 & 0 & 0 & 0 & 0 & 0 \\
0 & 0 & 1 - i & 1 + i & 0 & 0 & 0 & 0 \\
0 & 0 & 1 + i & 1 - i & 0 & 0 & 0 & 0 \\
\hline
0 & 0 & 0 & 0 & 1 & 0 & 0 & 0 \\
0 & 0 & 0 & 0 & 0 & 1 & 0 & 0 \\
0 & 0 & 0 & 0 & 0 & 0 & 1-i & 1+i \\
0 & 0 & 0 & 0 & 0 & 0 & 1+i & 1-i
\end{pmatrix}.
\end{align*}
\begin{align*}
CNOT \otimes I 
&= 
\begin{pmatrix}
1 & 0 & 0 & 0 \\
0 & 1 & 0 & 0 \\
0 & 0 & 0 & 1 \\
0 & 0 & 1 & 0 \\
\end{pmatrix}
\begin{pmatrix}
1 & 0 \\
0 & 1 \\
\end{pmatrix} \\
&= 
\begin{pmatrix}
1 & 0 & 0 & 0 & 0 & 0 & 0 & 0 \\
0 & 1 & 0 & 0 & 0 & 0 & 0 & 0 \\
0 & 0 & 1 & 0 & 0 & 0 & 0 & 0 \\
0 & 0 & 0 & 1 & 0 & 0 & 0 & 0 \\
\hline 
0 & 0 & 0 & 0 & 0 & 0 & 1 & 0 \\
0 & 0 & 0 & 0 & 0 & 0 & 0 & 1\\
0 & 0 & 0 & 0 & 1 & 0 & 0 & 0 \\
0 & 0 & 0 & 0 & 0 & 1 & 0 & 0 \\ 
\end{pmatrix}.
\end{align*}
\begin{align*}
I \otimes CV^{\dag} &= 
\begin{pmatrix}
1 & 0 \\
0 & 1
\end{pmatrix}
\otimes 
\begin{pmatrix}
1 & 0 & 0 & 0 \\
0 & 1 & 0 & 0 \\
0 & 0 & 1 +i & 1 - i \\
0 & 0 & 1 - i & 1 + i 
\end{pmatrix} \\
&= 
\begin{pmatrix}
1 & 0 & 0 & 0 & 0 & 0 & 0 & 0 \\
0 & 1 & 0 & 0 & 0 & 0 & 0 & 0 \\
0 & 0 & 1 + i & 1 - i & 0 & 0 & 0 & 0 \\
0 & 0 & 1 - i & 1 + i & 0 & 0 & 0 & 0 \\
\hline
0 & 0 & 0 & 0 & 1 & 0 & 0 & 0 \\
0 & 0 & 0 & 0 & 0 & 1 & 0 & 0 \\
0 & 0 & 0 & 0 & 0 & 0 & 1+i & 1-i \\
0 & 0 & 0 & 0 & 0 & 0 & 1-i & 1+i
\end{pmatrix}.
\end{align*}
Now
\begin{align*}
(I \otimes CV)(CNOT \otimes I)
&= 
\begin{pmatrix}
1 & 0 & 0 & 0 & 0 & 0 & 0 & 0 \\
0 & 1 & 0 & 0 & 0 & 0 & 0 & 0 \\
0 & 0 & 1 - i & 1 + i & 0 & 0 & 0 & 0 \\
0 & 0 & 1 + i & 1 - i & 0 & 0 & 0 & 0 \\
\hline
0 & 0 & 0 & 0 & 1 & 0 & 0 & 0 \\
0 & 0 & 0 & 0 & 0 & 1 & 0 & 0 \\
0 & 0 & 0 & 0 & 0 & 0 & 1-i & 1+i \\
0 & 0 & 0 & 0 & 0 & 0 & 1+i & 1-i
\end{pmatrix}
\begin{pmatrix}
1 & 0 & 0 & 0 & 0 & 0 & 0 & 0 \\
0 & 1 & 0 & 0 & 0 & 0 & 0 & 0 \\
0 & 0 & 1 & 0 & 0 & 0 & 0 & 0 \\
0 & 0 & 0 & 1 & 0 & 0 & 0 & 0 \\
\hline 
0 & 0 & 0 & 0 & 0 & 0 & 1 & 0 \\
0 & 0 & 0 & 0 & 0 & 0 & 0 & 1\\
0 & 0 & 0 & 0 & 1 & 0 & 0 & 0 \\
0 & 0 & 0 & 0 & 0 & 1 & 0 & 0 \\ 
\end{pmatrix} \\
&= 
\begin{pmatrix}
1 & 0 & 0      & 0    & 0   & 0    & 0 & 0 \\
0 & 1 & 0      & 0    & 0   & 0    & 0 & 0 \\
0 & 0 & 1 - i & 1+i & 0    & 0    & 0 & 0 \\
0 & 0 & 1+i  & 1-i  & 0    & 0    & 0 & 0 \\
0 & 0 & 0      & 0    & 0    & 0    & 1 & 0 \\
0 & 0 & 0      & 0    & 0    & 0    & 0 & 1 \\
0 & 0 & 0      & 0    & 1-i & 1+i & 0 & 0 \\
0 & 0 & 0      & 0    & 1+i & 1-i & 0 & 0 
\end{pmatrix}
\end{align*}
\begin{align*}
(CNOT \otimes I)(I \otimes CV)
&= 
\begin{pmatrix}
1 & 0 & 0 & 0 & 0 & 0 & 0 & 0 \\
0 & 1 & 0 & 0 & 0 & 0 & 0 & 0 \\
0 & 0 & 1 & 0 & 0 & 0 & 0 & 0 \\
0 & 0 & 0 & 1 & 0 & 0 & 0 & 0 \\
\hline 
0 & 0 & 0 & 0 & 0 & 0 & 1 & 0 \\
0 & 0 & 0 & 0 & 0 & 0 & 0 & 1\\
0 & 0 & 0 & 0 & 1 & 0 & 0 & 0 \\
0 & 0 & 0 & 0 & 0 & 1 & 0 & 0 \\ 
\end{pmatrix} 
\begin{pmatrix}
1 & 0 & 0 & 0 & 0 & 0 & 0 & 0 \\
0 & 1 & 0 & 0 & 0 & 0 & 0 & 0 \\
0 & 0 & 1 - i & 1 + i & 0 & 0 & 0 & 0 \\
0 & 0 & 1 + i & 1 - i & 0 & 0 & 0 & 0 \\
\hline
0 & 0 & 0 & 0 & 1 & 0 & 0 & 0 \\
0 & 0 & 0 & 0 & 0 & 1 & 0 & 0 \\
0 & 0 & 0 & 0 & 0 & 0 & 1-i & 1+i \\
0 & 0 & 0 & 0 & 0 & 0 & 1+i & 1-i
\end{pmatrix} \\
&= 
\begin{pmatrix}
1 & 0 & 0      & 0    & 0 & 0 & 0   & 0 \\
0 & 1 & 0      & 0    & 0 & 0 & 0   & 0 \\
0 & 0 & 1 - i & 1+i & 0 & 0 & 0   & 0 \\
0 & 0 & 1 +i & 1-i  & 0 & 0 & 0   & 0 \\
0 & 0 & 0     & 0     & 0 & 0 & 1-i & 1+ i \\
0 & 0 & 0     & 0     & 0 & 0 & 1+i & 1-i \\
0 & 0 & 0     & 0     & 1 & 0 & 0    & 0 \\
0 & 0 & 0     & 0     & 0 & 1 & 0    & 0 
\end{pmatrix}.
\end{align*}
\begin{align*}
(I \otimes CV)(CNOT \otimes I)(I \otimes CV^{\dag}) = \\ 
\begin{pmatrix}
1 & 0 & 0      & 0    & 0   & 0    & 0 & 0 \\
0 & 1 & 0      & 0    & 0   & 0    & 0 & 0 \\
0 & 0 & 1 - i & 0    & 0    & 0    & 0 & 0 \\
0 & 0 & 0      & 1-i & 0    & 0    & 0 & 0 \\
0 & 0 & 0      & 0    & 0    & 0    & 1 & 0 \\
0 & 0 & 0      & 0    & 0    & 0    & 0 & 1 \\
0 & 0 & 0      & 0    & 1-i & 1+i & 0 & 0 \\
0 & 0 & 0      & 0    & 1+i & 1-i & 0 & 0 
\end{pmatrix}
\begin{pmatrix}
1 & 0 & 0 & 0 & 0 & 0 & 0 & 0 \\
0 & 1 & 0 & 0 & 0 & 0 & 0 & 0 \\
0 & 0 & 1 + i & 1 - i & 0 & 0 & 0 & 0 \\
0 & 0 & 1 - i & 1 + i & 0 & 0 & 0 & 0 \\
\hline
0 & 0 & 0 & 0 & 1 & 0 & 0 & 0 \\
0 & 0 & 0 & 0 & 0 & 1 & 0 & 0 \\
0 & 0 & 0 & 0 & 0 & 0 & 1+i & 1-i \\
0 & 0 & 0 & 0 & 0 & 0 & 1-i & 1+i
\end{pmatrix} \\
=
\begin{pmatrix}
1 & 0 & 0    & 0      & 0    & 0 & 0     & 0 \\
0 & 1 & 0    & 0      & 0    & 0 & 0     & 0 \\
0 & 0 & 2    & -2i    & 0    & 0 & 0     & 0 \\
0 & 0 & -2i & 2       & 0    & 0 & 0     & 0 \\
0 & 0 & 0    & 0       & 0    & 0 & 1+i & 1-i \\
0 & 0 & 0    & 0       & 0    & 0 & 1-i  & 1+ i \\
0 & 0 & 0    & 0       & 1-i & 1+i & 0 & 0 \\
0 & 0 & 0    & 0       & 1+i & 1-i & 0 & 0 
\end{pmatrix}.
\end{align*}
Finally,
\begin{align*}
(I \otimes CV)(CNOT \otimes I)(I \otimes CV^{\dag})(CNOT \otimes I)(I \otimes CV) \\
=
\begin{pmatrix}
1 & 0 & 0    & 0      & 0    & 0 & 0     & 0 \\
0 & 1 & 0    & 0      & 0    & 0 & 0     & 0 \\
0 & 0 & 2    & -2i    & 0    & 0 & 0     & 0 \\
0 & 0 & -2i & 2       & 0    & 0 & 0     & 0 \\
0 & 0 & 0    & 0       & 0    & 0 & 1+i & 1-i \\
0 & 0 & 0    & 0       & 0    & 0 & 1-i  & 1+ i \\
0 & 0 & 0    & 0       & 1-i & 1+i & 0 & 0 \\
0 & 0 & 0    & 0       & 1+i & 1-i & 0 & 0 
\end{pmatrix}
\begin{pmatrix}
1 & 0 & 0      & 0    & 0 & 0 & 0   & 0 \\
0 & 1 & 0      & 0    & 0 & 0 & 0   & 0 \\
0 & 0 & 1 - i & 1+i & 0 & 0 & 0   & 0 \\
0 & 0 & 1 +i & 1-i  & 0 & 0 & 0   & 0 \\
0 & 0 & 0     & 0     & 0 & 0 & 1-i & 1+ i \\
0 & 0 & 0     & 0     & 0 & 0 & 1+i & 1-i \\
0 & 0 & 0     & 0     & 1 & 0 & 0    & 0 \\
0 & 0 & 0     & 0     & 0 & 1 & 0    & 0 
\end{pmatrix} =\\
\begin{pmatrix}
1 & 0 & 0                        & 0                      & 0    & 0    & 0 & 0 \\
0 & 1 & 0                        & 0                      & 0    & 0    & 0 & 0 \\
0 & 0 & 2(1-i) -2i(1+i)    & 2(1+i) -2i(1-i)  & 0    & 0    & 0 & 0 \\
0 & 0 & -2i(1-i) +2(1+i) & -2i(1+i)+2(1-i) & 0    & 0    & 0 & 0 \\
0 & 0 & 0                        & 0                      & 1+i & 1-i & 0 & 0 \\
0 & 0 & 0                        & 0                      & 1-i & 1+i & 0 & 0 \\
0 & 0 & 0      			       & 0                      & 0    & 0    & (1-i)^2 + (1+i)^2 & 2(1-i)(1+i) \\
0 & 0 & 0  					   & 0 				   	   & 0    & 0    & 2(1+i)(1-i)   & (1+i)^2 + (1-i)^2 \\
\end{pmatrix} \\
= 
\begin{pmatrix}
1 & 0 & 0 & 0   & 0 & 0 & 0 & 0 \\
0 & 1 & 0 & 0   & 0 & 0 & 0 & 0 \\ 2-2i - 2i +2 = 4 - 4i
0 & 0 & 4(1 -i) & 0 & 0 & 0 & 0 & 0 \\
\end{pmatrix}
\end{align*}
\section*{Problems not solved}
\begin{itemize}
\item QCE 7.1 
\item (Ask about QCE 7.9)
\item (Ask about QCE 7.14)
\item (Ask about QCE 8.1)
\item QCE 8.12
\end{itemize}

\end{document}