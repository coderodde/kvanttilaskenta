\documentclass[10pt]{article}

\usepackage[english]{babel}
\usepackage[utf8x]{inputenc}
\usepackage{amsmath}
\usepackage{amssymb}
\usepackage{amsfonts}
\usepackage{graphicx}
\usepackage[ruled,linesnumbered,noend]{algorithm2e}
\usepackage{empheq}
\usepackage{float}
\usepackage{enumitem}
\usepackage{tikz}
\usepackage[colorlinks=true,urlcolor=blue]{hyperref}
\usepackage{braket}
\usepackage{gensymb}

\title{Kvanttilaskenta, kevät 2015 -- Viikko 2}
\author{Rodion ``rodde'' Efremov}

\begin{document}
 \maketitle

\section*{edx Problem 1}
Let 
\[
U =
\begin{pmatrix}
-2i & 5i \\
5 & 1 - i 
\end{pmatrix}.
\]
Now
\[
U^{\dag} = 
\begin{pmatrix}
2i & 5 \\
-5i & 1 + i
\end{pmatrix}.
\]

\section*{edx Problem 2}
All unitary matrices are self-inverse.

\section*{edx Problem 3}
We are given a $U$ that maps $\ket{0}$ to $\frac{1}{\sqrt{2}} \ket{0} + \frac{1+i}{2} \ket{1}$ and $\ket{1}$ to $\frac{1 - i}{2} \ket{0} - \frac{1}{\sqrt{2}} \ket{1}$. So we have that
\begin{align*}
U\ket{0} &= 
\begin{pmatrix}
a & b \\
c & d
\end{pmatrix}
\begin{pmatrix}
1 \\
0
\end{pmatrix} \\
&= 
\begin{pmatrix}
a \\
c
\end{pmatrix} \\
&= 
\begin{pmatrix}
\frac{1}{\sqrt{2}} \\
\frac{1 + i}{2}
\end{pmatrix}.
\end{align*}
Also, we have that
\begin{align*}
U\ket{1} &= 
\begin{pmatrix}
a & b \\
c & d
\end{pmatrix}
\begin{pmatrix}
0 \\
1
\end{pmatrix} \\
&= 
\begin{pmatrix}
b \\
d
\end{pmatrix} \\
&= 
\begin{pmatrix}
1-i \\
-\frac{1}{\sqrt{2}}
\end{pmatrix}.
\end{align*}
It follows immediately that
\[
U = 
\begin{pmatrix}
\frac{1}{\sqrt{2}} & 1 - i \\
\frac{1 + i}{2} & -\frac{1}{\sqrt{2}}
\end{pmatrix}.
\]

\section*{edx Problem 4}
Suppose we have a one-qubit unitary $U$ that maps $\ket{0}$ to $\frac{-3}{5} \ket{0} + \frac{4i}{5} \ket{1}$ and $\ket{+}$ to $\frac{-3-4i}{5\sqrt{2}} \ket{0} + \frac{3+4i}{5\sqrt{2}} \ket{1}$.

We have that
\begin{align*}
U\ket{0} &= 
\begin{pmatrix}
a & b \\
c & d
\end{pmatrix}
\begin{pmatrix}
1 \\
0
\end{pmatrix} \\
&= 
\begin{pmatrix}
a \\
c
\end{pmatrix} \\
&= 
\begin{pmatrix}
\frac{-3}{5} \\
\frac{4i}{5}
\end{pmatrix}.
\end{align*}
As $\ket{+} = (\frac{1}{\sqrt{2}} \quad \frac{1}{\sqrt{2}})^T$, we have that
\begin{align*}
U\ket{+} &= 
\begin{pmatrix}
a & b \\
c & d
\end{pmatrix}
\begin{pmatrix}
\frac{1}{\sqrt{2}} \\
\frac{1}{\sqrt{2}}
\end{pmatrix} \\
&= 
\begin{pmatrix}
\frac{1}{\sqrt{2}} (a + b) \\
\frac{1}{\sqrt{2}} (c + d) \\
\end{pmatrix} \\
&= 
\begin{pmatrix}
\frac{1}{\sqrt{2}} (-\frac{3}{5} + b) \\
\frac{1}{\sqrt{2}} (\frac{4i}{5} + d)
\end{pmatrix} \\
&=
\begin{pmatrix}
\frac{-3-4i}{5\sqrt{2}} \\
\frac{3+4i}{5\sqrt{2}}
\end{pmatrix}.
\end{align*}
Now for $b$ we have an equality
\[
-\frac{3}{5} + b = \frac{-3-4i}{5},
\]
which has solution $b = -\frac{4i}{5}$.
For $d$ we have an equality
\[
\frac{4i}{5} + d = \frac{3+4i}{5},
\]
which has solution $d = \frac{3}{5}$.

So the matrix in question is
\[
U = 
\begin{pmatrix}
-\frac{3}{5} & -\frac{4i}{5} \\
\frac{4i}{5} & \frac{3}{5}
\end{pmatrix}.
\]

\section*{edx Problem 4 once again - Thanks, Tomi!}
$\frac{1}{\sqrt{2}} (\ket{00} + \ket{11})$.

\section*{edx Problem 5}
$\frac{1}{\sqrt{2}} (\ket{00} - \ket{11})$.

\section*{edx Problem 6}
True.

\section*{edx Problem 5}
What is $ZX$ applied to $\ket{0}$?
We are given Pauli operators
\[
Z = \begin{pmatrix}
1 & 0 \\
0 & -1 
\end{pmatrix}, \quad
X = \begin{pmatrix}
0 & 1 \\
1 & 0
\end{pmatrix},
\]
so 
\begin{align*}
ZX\ket{0} &= \begin{pmatrix}
1 & 0 \\
0 & -1
\end{pmatrix}
\begin{pmatrix}
0 & 1 \\
1 & 0
\end{pmatrix}
\begin{pmatrix}
1 \\
0
\end{pmatrix} \\
&= \begin{pmatrix}
0 & 1 \\
-1 & 0
\end{pmatrix}
\begin{pmatrix}
1 \\
0
\end{pmatrix} \\
&= \begin{pmatrix}
0 \\
-1
\end{pmatrix} \\
&= -\ket{1}.
\end{align*}

\section*{edx Problem 6}
What is $ZX$ applied to $H\ket{0}$?

From the exerices above we have that
\[
ZX = \begin{pmatrix}
0 & 1 \\
-1 & 0
\end{pmatrix}.
\]
Also $H\ket{0} = \frac{1}{\sqrt{2}} \ket{0} + \frac{1}{\sqrt{2}} \ket{1}$, so
\begin{align*}
ZXH\ket{0} &= \begin{pmatrix}
0 & 1 \\
-1 & 0
\end{pmatrix}
\begin{pmatrix}
\frac{1}{\sqrt{2}} \\
\frac{1}{\sqrt{2}} 
\end{pmatrix} \\
&= \begin{pmatrix}
\frac{1}{\sqrt{2}} \\
-\frac{1}{\sqrt{2}}
\end{pmatrix} \\
&= \ket{-}.
\end{align*}

\section*{edx Problem 7}
We are given a qubit $\ket{\psi} = \alpha \ket{0} + \beta \ket{1}$ and we know that $|\alpha|^2 = \frac{2}{9}$ so $|\beta|^2 = \frac{7}{9}$. Also we know that
\[
H = \frac{1}{\sqrt{2}} \begin{pmatrix}
1 & 1 \\
1 & -1
\end{pmatrix},
\]
so 
\begin{align*}
H\ket{\psi} &= \frac{1}{\sqrt{2}} \begin{pmatrix}
1 & 1 \\
1 & -1
\end{pmatrix} \begin{pmatrix}
\alpha \\
\beta
\end{pmatrix} \\
&= \frac{1}{\sqrt{2}} \begin{pmatrix}
\alpha + \beta \\
\alpha - \beta
\end{pmatrix}
\end{align*}
Also we know that $\ket{+} = \frac{1}{\sqrt{2}} (\ket{0} + \ket{1})$. Now the probability of measuring $+$ is 
\begin{align*}
\cos^2 \theta &= |\braket{H\psi | +}|^2 \\
					   &= \Bigg|\frac{1}{\sqrt{2}}(\alpha + \beta)(\frac{1}{\sqrt{2}}) + \frac{1}{\sqrt{2}}(\alpha - \beta)(\frac{1}{\sqrt{2}}) \Bigg|^2 \\
					   &= \Bigg| \frac{1}{2} (\alpha + \beta) + \frac{1}{2} (\alpha - \beta) \Bigg|^2 \\
					   &= | \alpha |^2 \\
					   &= \frac{2}{9}.
\end{align*}

\section*{edx Problem 8}
All pairs commute except CNOT and X applied to the target qubit.

\section*{edx Problem 9}
Apply the 3rd (last) circuit from above.

\section*{edx Problem 10}
(a) The resulting state ain't entangled.
(b) First qubit: $b\ket{0} + a\ket{1}$, second cubit: $\ket{0}$.

\section*{edx Problem 11}
No circuit exists by no cloning theorem.

\section*{edx Problem 12}
I think in general case the correct alternative is $[0, \frac{\pi}{2}]$, yet unitary matrices preserve angles, so for any unitary $U$ $U\ket{\psi}$ $U\ket{\psi'}$ have the same angle as $\ket{\psi}$ and $\ket{\psi'}$.

\section*{edx Problem 13}
When Alice's outcome was 0, apply $I$. When Alice's outcome was 1, apply $Z$.

\section*{QCE 5.1}
Consider the following state vector:
\[
\ket{\psi} = \sqrt{\frac{5}{6}} \ket{0} + \frac{1}{\sqrt{6}} \ket{1}.
\]
(A) Is the state normalized?
As
\[
\Bigg( \sqrt{\frac{5}{6}} \Bigg)^2 + \Bigg( \frac{1}{\sqrt{6}} \Bigg)^2 = \frac{5}{6} + \frac{1}{6} = 1, 
\]
the state vector is normalized.

(B) What is the probability that the system is found to be in state $\ket{0}$ if $Z$ is measured? 

After applying the Z-gate, we obtain
\begin{align*}
Z\ket{\psi} &= 
\begin{pmatrix}
1 & 0 \\
0 & -1
\end{pmatrix}
\begin{pmatrix}
\sqrt{\frac{5}{6}} \\
\frac{1}{\sqrt{6}}
\end{pmatrix} \\
&= 
\begin{pmatrix}
\sqrt{\frac{5}{6}} \\
-\frac{1}{\sqrt{6}}
\end{pmatrix},
\end{align*}
which implies that the probability in question is $\frac{5}{5}$.

Lets do this the adult way:
As $P_0 = \ket{0}\bra{0}$ and $P_1 = \ket{1} \bra{1}$, we have that
\[
P_0 = \begin{pmatrix}
\braket{0|P_0|0} & \braket{0|P_0|1} \\
\braket{1|P_0|0} & \braket{1|P_0|1} 
\end{pmatrix}
= \begin{pmatrix}
1 & 0 \\
0 & 0
\end{pmatrix}
\]
and
\[
P_1 = \begin{pmatrix}
\braket{0|P_1|0} & \braket{0|P_1|1} \\
\braket{1|P_1|0} & \braket{1|P_1|1} 
\end{pmatrix}
= \begin{pmatrix}
0 & 0 \\
0 & 1
\end{pmatrix}
\]
Next we need the density operator, which is given by
\begin{align*}
\rho &= \ket{\psi} \bra{\psi} \\
        &= \Bigg( \sqrt{\frac{5}{6}} \ket{0} + \frac{1}{\sqrt{6}} \ket{1} \Bigg) \Bigg( \sqrt{\frac{5}{6}} \bra{0} + \frac{1}{\sqrt{6}} \bra{1} \Bigg) \\
        &= \frac{5}{6} \ket{0} \bra{0} + \frac{\sqrt{5}}{6} \ket{0} \bra{1} + \frac{\sqrt{5}}{6} \ket{1} \bra{0} + \frac{1}{6} \ket{1} \bra{1},
\end{align*}
so the density matrix in the $\{ \ket{0}, \ket{1} \}$ basis is
\[
\begin{pmatrix}
\frac{5}{6} & \frac{\sqrt{5}}{6} \\
\frac{\sqrt{5}}{6} & \frac{1}{6}
\end{pmatrix}.
\]
Now the probability of finding the system in state $\ket{0}$ is
\[
p(0) = Tr(P_0 \rho) = Tr \begin{bmatrix}
\begin{pmatrix}
1 & 0 \\
0 & 0
\end{pmatrix}
\begin{pmatrix}
\frac{5}{6} & \frac{\sqrt{5}}{6} \\
\frac{\sqrt{5}}{6} & \frac{1}{6}
\end{pmatrix}
\end{bmatrix}
= Tr \begin{pmatrix}
\frac{5}{6} & \frac{\sqrt{5}}{6} \\
0 & 0
\end{pmatrix}
= \frac{5}{6}.
\]

(C) Write down the density operator. See above.

(D) Find the density matrix in the $\{ \ket{0}, \ket{1} \}$ basis, and show that $Tr(\rho) = 1$. See above.

\section*{QCE 5.2}
\[
\ket{\psi} = \begin{pmatrix}
\cos \theta \\
i \sin \theta
\end{pmatrix},
\]
so 
\[
|\braket{\psi|\psi}| = |(\cos \theta + i \sin \theta)( \cos \theta - i \sin \theta )| = |\cos^2 \theta + \sin^2 \theta| = |1| = 1
\]
and the state is normalized.
Now 
\begin{align*}
\rho &= \ket{\psi} \bra{\psi} \\
        &= ( \cos \theta \ket{0} + i \sin \theta \ket{1} ) ( \cos \theta \bra{0} + i \sin \theta \bra{1} ) \\
        &= \cos^2 \theta \ket{0} \bra{0} + i \sin \theta \cos \theta \ket{0} \bra{1} + i \sin \theta \cos \theta \ket{1} \bra{0} - \sin^2 \theta \ket{1} \bra{1} \\
        &= \begin{pmatrix}
        \cos^2 \theta & i \sin \theta \cos \theta \\
        i \sin \theta \cos \theta & \sin^2 \theta
        \end{pmatrix}.
\end{align*}
Obviously, $Tr(\rho) = \cos^2 \theta + \sin^2 \theta = 1$. 
Also 
\begin{align*}
\rho^{\dag} &= \begin{pmatrix}
        \cos^2 \theta & i \sin \theta \cos \theta \\
        i \sin \theta \cos \theta & \sin^2 \theta
        \end{pmatrix}^{\dag} \\
        &= 
        \begin{pmatrix}
        \cos^2 \theta & i \sin \theta \cos \theta \\
        i \sin \theta \cos \theta & \sin^2 \theta
        \end{pmatrix}^{\star} \\
        &= 
        \begin{pmatrix}
        \cos^2 \theta & -i \sin \theta \cos \theta \\
        -i \sin \theta \cos \theta & \sin^2 \theta
        \end{pmatrix} \\
        &\neq \rho,
\end{align*}
so the operator is not Hermitian, and, thus, not a density operator.

\section*{QCE 5.3}
Let
\[
\ket{\psi} = \sqrt{\frac{3}{7}} \ket{0} + \frac{2}{\sqrt{7}} \ket{1}.
\]
(A) The density matrix in the $\{ \ket{0}, \ket{1} \}$ basis is
\begin{align*}
\rho &= \ket{\psi} \bra{\psi} \\
        &= (\sqrt{\frac{3}{7}} \ket{0} + \frac{2}{\sqrt{7}} \ket{1})( \sqrt{\frac{3}{7}} \bra{0} + \frac{2}{\sqrt{7}} \bra{1}) \\
        &= \frac{3}{7} \ket{0}\bra{0} + \frac{2\sqrt{3}}{7} \ket{0}\bra{1} + \frac{2\sqrt{3}}{7} \ket{1}\bra{0} + \frac{4}{7} \ket{1}\bra{1} \\
        &=  \begin{pmatrix}
        \frac{3}{7} & \frac{2\sqrt{3}}{7} \\
        \frac{2\sqrt{3}}{7} & \frac{4}{7}
        \end{pmatrix}.
\end{align*}
Next we need $\rho^2$ which is given by
\begin{align*}
& \begin{pmatrix}
     \frac{3}{7} & \frac{2\sqrt{3}}{7} \\
     \frac{2\sqrt{3}}{7} & \frac{4}{7}
\end{pmatrix}
\begin{pmatrix}
     \frac{3}{7} & \frac{2\sqrt{3}}{7} \\
     \frac{2\sqrt{3}}{7} & \frac{4}{7}
\end{pmatrix} = \\
& \frac{1}{49} \begin{pmatrix}
3 & 2\sqrt{3} \\
2\sqrt{3} & 4
\end{pmatrix}
\begin{pmatrix}
3 & 2\sqrt{3} \\
2\sqrt{3} & 4
\end{pmatrix} = \\
& \frac{1}{49}
\begin{pmatrix}
9 + 12 & 6\sqrt{3} + 8\sqrt{3} \\
6\sqrt{3} + 8\sqrt{3} & 12 + 16
\end{pmatrix} = \\
& \frac{1}{49}
\begin{pmatrix}
21 & 14\sqrt{3} \\
14 \sqrt{3}  & 28
\end{pmatrix}.
\end{align*}
As $Tr(\rho^2) = 1$, this is a pure state.

(C) Write down the density matrix in the $\{ \ket{+}, \ket{-} \}$ basis, show that $Tr(\rho) = 1$ still holds, and determine if you still obtain the same result as in part (b).

Here we have
\begin{align*}
\ket{\psi} &= \sqrt{\frac{3}{7}} \ket{0} + \frac{2}{\sqrt{7}} \ket{1} \\
                &= \sqrt{\frac{3}{7}} \frac{1}{\sqrt{2}} (\ket{+} + \ket{-}) + \frac{2}{\sqrt{7}} \frac{1}{\sqrt{2}} (\ket{+} - \ket{-}) \\
                &= \sqrt{\frac{3}{14}} (\ket{+} + \ket{-}) + \sqrt{\frac{4}{14}}(\ket{+} - \ket{-}) \\
                &= \Bigg( \sqrt{\frac{3}{14}} + \sqrt{\frac{4}{14}} \Bigg) \ket{+} + \Bigg( \sqrt{\frac{3}{14}}- \sqrt{\frac{4}{14}} \Bigg) \ket{-} 
\end{align*}
Now
\begin{align*}
\rho &= \ket{\psi} \bra{\psi} \\
        &= \Bigg( \Bigg( \sqrt{\frac{3}{14}} + \sqrt{\frac{4}{14}} \Bigg) \ket{+} + \Bigg( \sqrt{\frac{3}{14}}- \sqrt{\frac{4}{14}} \Bigg) \ket{-} \Bigg) \Bigg( \Bigg( \sqrt{\frac{3}{14}} + \sqrt{\frac{4}{14}} \Bigg) \bra{+} + \Bigg( \sqrt{\frac{3}{14}}- \sqrt{\frac{4}{14}} \Bigg) \bra{-} \Bigg) \\
        &= 
        \Bigg( \sqrt{\frac{3}{14}} + \sqrt{\frac{4}{14}} \Bigg)^2 \ket{+} \bra{+} \\
        &+ \Bigg( \sqrt{\frac{3}{14}} + \sqrt{\frac{4}{14}} \Bigg) \Bigg( \sqrt{\frac{3}{14}}- \sqrt{\frac{4}{14}} \Bigg) \ket{+} \bra{-} \\
        &+ \Bigg( \sqrt{\frac{3}{14}} + \sqrt{\frac{4}{14}} \Bigg) \Bigg( \sqrt{\frac{3}{14}}- \sqrt{\frac{4}{14}} \Bigg) \ket{-} \bra{+} \\
        &+ \Bigg( \sqrt{\frac{3}{14}} - \sqrt{\frac{4}{14}} \Bigg)^2 \ket{-} \bra{-} \\
        &= 
        \Bigg( \sqrt{\frac{3}{14}} + \sqrt{\frac{4}{14}} \Bigg)^2 \ket{+} \bra{+} \\
        &+ \Bigg( \frac{3}{14} - \frac{4}{14} \Bigg) \ket{+} \bra{-} \\
        &+ \Bigg( \frac{3}{14} - \frac{4}{14} \Bigg) \ket{-} \bra{+} \\
        &+ \Bigg( \sqrt{\frac{3}{14}} - \sqrt{\frac{4}{14}} \Bigg)^2 \ket{-} \bra{-} \\
        &= \Bigg( \sqrt{\frac{3}{14}} + \sqrt{\frac{4}{14}} \Bigg)^2 \ket{+} \bra{+} 
        -\frac{1}{14} \ket{+} \bra{-} 
        -\frac{1}{14} \ket{-} \bra{+} 
        + \Bigg( \sqrt{\frac{3}{14}} - \sqrt{\frac{4}{14}} \Bigg)^2 \ket{-} \bra{-} \\
        &= \begin{pmatrix}
        \Bigg( \sqrt{\frac{3}{14}} + \sqrt{\frac{4}{14}} \Bigg)^2 & -\frac{1}{14} \\
        -\frac{1}{14} & \Bigg( \sqrt{\frac{3}{14}} - \sqrt{\frac{4}{14}} \Bigg)^2
        \end{pmatrix}.
\end{align*}
Now
\begin{align*}
Tr(\rho) &= \Bigg( \sqrt{\frac{3}{14}} + \sqrt{\frac{4}{14}} \Bigg)^2 + \Bigg( \sqrt{\frac{3}{14}} - \sqrt{\frac{4}{14}} \Bigg)^2 \\
             &= \frac{3}{14} + \frac{4}{14} + 2\frac{\sqrt{12}}{14} + \frac{3}{14} + \frac{4}{14} -2 \frac{\sqrt{12}}{14} \\
             &= 1.
\end{align*}
Also
\begin{align*}
\rho^2 &= \begin{pmatrix}
        \Bigg( \sqrt{\frac{3}{14}} + \sqrt{\frac{4}{14}} \Bigg)^2 & -\frac{1}{14} \\
        -\frac{1}{14} & \Bigg( \sqrt{\frac{3}{14}} - \sqrt{\frac{4}{14}} \Bigg)^2
        \end{pmatrix}
        \begin{pmatrix}
        \Bigg( \sqrt{\frac{3}{14}} + \sqrt{\frac{4}{14}} \Bigg)^2 & -\frac{1}{14} \\
        -\frac{1}{14} & \Bigg( \sqrt{\frac{3}{14}} - \sqrt{\frac{4}{14}} \Bigg)^2
        \end{pmatrix} \\
        &= \begin{pmatrix}
        \Bigg( \sqrt{\frac{3}{14}} + \sqrt{\frac{4}{14}} \Bigg)^4 + \frac{1}{196} & \dots \\
        \dots & \Bigg( \sqrt{\frac{3}{14}} - \sqrt{\frac{4}{14}} \Bigg)^4 + \frac{1}{196}
        \end{pmatrix}.
\end{align*}
According to Wolframalpha, $Tr(\rho^2) = 1$, so this state is pure.

\section*{QCE 5.4}
Let 
\[
\ket{\psi} = \sqrt{\frac{2}{3}} \ket{0} + \frac{1}{\sqrt{3}} \ket{1}.
\]
Now 
\begin{align*}
\rho &= \ket{\psi} \bra{\psi} \\
        &= \Bigg( \sqrt{\frac{2}{3}} \ket{0} + \frac{1}{\sqrt{3}} \ket{1} \Bigg)\Bigg( \sqrt{\frac{2}{3}} \bra{0} + \frac{1}{\sqrt{3}} \bra{1} \Bigg) \\
        &= \frac{2}{3} \ket{0} \bra{0} + \frac{\sqrt{2}}{3} \ket{0} \bra{1} + \frac{\sqrt{2}}{3} \ket{1} \bra{0} + \frac{1}{3} \ket{1} \bra{1} \\
        &= \begin{pmatrix}
        \frac{2}{3} & \frac{\sqrt{2}}{3} \\
        \frac{\sqrt{2}}{3} & \frac{1}{3}
        \end{pmatrix}.
\end{align*}
It is obvious that $Tr(\rho) = 1$.
\begin{align*}
\rho^2 &= 
\begin{pmatrix}
        \frac{2}{3} & \frac{\sqrt{2}}{3} \\
        \frac{\sqrt{2}}{3} & \frac{1}{3}
\end{pmatrix}
\begin{pmatrix}
        \frac{2}{3} & \frac{\sqrt{2}}{3} \\
        \frac{\sqrt{2}}{3} & \frac{1}{3}
\end{pmatrix} \\
    &= \begin{pmatrix}
    \frac{4}{9} + \frac{2}{9} & \frac{2\sqrt{2}}{9} + \frac{\sqrt{2}}{9} \\
    \frac{2\sqrt{2}}{9} + \frac{\sqrt{2}}{9} & \frac{2}{9} + \frac{1}{9} 
\end{pmatrix}  \\
    &= \begin{pmatrix}
    \frac{6}{9}             & \frac{3\sqrt{2}}{9} \\
    \frac{3\sqrt{2}}{9} & \frac{3}{9}
    \end{pmatrix}.
\end{align*}
$Tr(\rho^2) = 1$ so the state is pure.

In order to compute $\braket{X}$ we can fall down to equation $\braket{X} = Tr(\rho X)$. Now
\begin{align*}
\rho X &= 
\begin{pmatrix}
        \frac{2}{3} & \frac{\sqrt{2}}{3} \\
        \frac{\sqrt{2}}{3} & \frac{1}{3}
\end{pmatrix}
\begin{pmatrix}
         0 & 1 \\
         1 & 0
\end{pmatrix} \\
    &= 
    \begin{pmatrix}
    \frac{\sqrt{2}}{3} & \frac{2}{3} \\
    \frac{1}{3}           & \frac{\sqrt{2}}{3}
    \end{pmatrix},
\end{align*}
so $\braket{X} = Tr(\rho X) = \frac{2\sqrt{2}}{3}$.

\section*{QCE 5.5}
Suppose that
\[
\rho = \begin{pmatrix}
\frac{1}{3}  & \frac{i}{4} \\
\frac{-i}{4} & \frac{2}{3}
\end{pmatrix}.
\]
As $Tr(\rho) = 1$ and $\rho = \rho^{\dag}$, $\rho$ is a valid density matrix. Now
\begin{align*}
\rho^2 &= 
\begin{pmatrix}
   \frac{1}{3}  & \frac{i}{4} \\
   \frac{-i}{4} & \frac{2}{3}
\end{pmatrix}
\begin{pmatrix}
   \frac{1}{3}  & \frac{i}{4} \\
   \frac{-i}{4} & \frac{2}{3}
\end{pmatrix} \\
      &= \begin{pmatrix}
         \frac{1}{9} + \frac{1}{16}    &  \frac{i}{12} + \frac{2i}{12} \\
         -\frac{i}{12} - \frac{2i}{12} &  \frac{1}{16} + \frac{4}{9}
      \end{pmatrix} \\
      &= \begin{pmatrix}
      \frac{25}{144} &  \frac{3i}{12} \\
      \frac{-3i}{12}  & \frac{73}{144}
      \end{pmatrix}.
\end{align*}
Now, it is obvious that this is not a pure state as $Tr(\rho^2) \neq 1$.

\section*{QCE 5.6}
Let 
\[
\rho = \frac{1}{5} \begin{pmatrix}
3       & 1 - i \\
1 + i & 2
\end{pmatrix}.
\]
Now
\begin{align*}
\rho^2 &= \frac{1}{25} \begin{pmatrix}
3       & 1 - i \\
1 + i & 2
\end{pmatrix}
\begin{pmatrix}
3       & 1 - i \\
1 + i & 2
\end{pmatrix} \\
           &= \frac{1}{25} \begin{pmatrix}
              9 + (1 - i)(1 + i)   &   3 - 3i + 2 - 2i \\
              3 + 3i + 2 + 2i     &   (1 - i)(1 + i) + 4
           \end{pmatrix} \\
           &= \begin{pmatrix}
           9 + 1 - i^2 & 5 - 5i \\
           5 + 5i         & 1 - i^2 + 4
           \end{pmatrix} \\
           &= \begin{pmatrix}
			11        &  5 - 5i \\
			5 + 5i & 6
           \end{pmatrix},
\end{align*}
so the state is mixed as $Tr(\rho^2) = \frac{17}{25} \neq 1$. Next, let us compute $\braket{X}, \braket{Y}, \braket{Z}$:
\begin{align*}
\rho X &= \frac{1}{5} \begin{pmatrix}
         3      & 1 - i \\
         1 + i & 2
     \end{pmatrix}
    \begin{pmatrix}
        0 & 1 \\
        1 & 0 
    \end{pmatrix} \\
&= \frac{1}{5} \begin{pmatrix}
1 - i   & 3 \\
2        & 1 + i
\end{pmatrix},
\end{align*}
so $\braket{X} = Tr(\rho X) = \frac{2}{5}$.
\begin{align*}
\rho Y &= \frac{1}{5} \begin{pmatrix}
         3      & 1 - i \\
         1 + i & 2
     \end{pmatrix}
    \begin{pmatrix}
        0 & -i \\
        i & 0 
    \end{pmatrix} \\
&= \frac{1}{5} 
\begin{pmatrix}
i - i^2 & -3i \\
2i        & -i -i^2 
\end{pmatrix} \\
&= \frac{1}{5} \begin{pmatrix}
1 + i & -3i \\
2i     & 1 - i
\end{pmatrix},
\end{align*}
so $\braket{Y} = Tr(\rho Y) = \frac{2}{5}$.
\begin{align*}
\rho Z &= \frac{1}{5} \begin{pmatrix}
         3      & 1 - i \\
         1 + i & 2
     \end{pmatrix}
    \begin{pmatrix}
        1 & 0 \\
        0 & -1 
    \end{pmatrix} \\
&= \frac{1}{5} 
\begin{pmatrix}
  3      & i - 1 \\
  1 + i & -2
\end{pmatrix},
\end{align*}
so $\braket{Z} = Tr(\rho Z) = \frac{1}{5}$.

\section*{QCE 5.7}
Let 
\[
\ket{\psi}= \frac{2}{\sqrt{5}} \ket{0} + \frac{1}{\sqrt{5}} \ket{1}.
\]
Now 
\begin{align*}
\rho_{\psi} &= \ket{\psi} \bra{\psi} \\
                  &= (\frac{2}{\sqrt{5}} \ket{0} + \frac{1}{\sqrt{5}} \ket{1}) (\frac{2}{\sqrt{5}} \bra{0} + \frac{1}{\sqrt{5}} \bra{1}) \\
                  &= \frac{4}{5} \ket{0} \bra{0} + \frac{2}{5} \ket{0} \bra{1} + \frac{2}{5} \ket{1} \bra{0} + \frac{1}{5} \ket{1} \bra{1} \\
                  &= \begin{pmatrix}
                  \frac{4}{5} & \frac{2}{5} \\
                  \frac{2}{5} & \frac{1}{5}
                  \end{pmatrix} \\
     &= \frac{1}{5}
     \begin{pmatrix}
	4 & 2 \\
	2 & 1     
     \end{pmatrix},
\end{align*}
so 
\begin{align*}
\rho_{\psi}^2 &= \frac{1}{25}
\begin{pmatrix}
				4 & 2 \\
				2 & 1
\end{pmatrix}
\begin{pmatrix}
				4 & 2 \\
				2 & 1
\end{pmatrix} \\
&= \frac{1}{25} \begin{pmatrix}
20 & 10 \\
10 & 5
\end{pmatrix},
\end{align*}
so $Tr(\rho_{\psi}^2) = 1$ and the state is pure.

Let 
\[
\ket{\phi} = \frac{1}{\sqrt{2}} \ket{0} + \frac{1}{\sqrt{2}} \ket{1}.
\]
Now
\begin{align*}
\rho_{\phi} &= \ket{\phi} \bra{\phi} \\
                  &= (\frac{1}{\sqrt{2}} \ket{0} + \frac{1}{\sqrt{2}} \ket{1})  (\frac{1}{\sqrt{2}} \bra{0} + \frac{1}{\sqrt{2}} \bra{1}) \\
                  &= \frac{1}{2} \ket{0} \bra{0} + \frac{1}{2} \ket{0} \bra{1} + \frac{1}{2} \ket{1} \bra{0} + \frac{1}{2} \ket{1} \bra{1} \\
                  &= \frac{1}{2} 
                  \begin{pmatrix}
                     1 & 1 \\
                     1 & 1
                  \end{pmatrix},
\end{align*}
so 
\begin{align*}
\rho_{\phi}^2 &= \frac{1}{4} 
\begin{pmatrix}
1 &  1 \\
1 & 1 
\end{pmatrix}
\begin{pmatrix}
1 &  1 \\
1 & 1 
\end{pmatrix} \\
&= \frac{1}{4} 
\begin{pmatrix}
2 & 2 \\
2 & 2
\end{pmatrix},
\end{align*}
so $Tr(\rho_{\phi}^2) = 1$ and the state is pure.

The density operator for the ensemble is given by 
\begin{align*}
\rho &= \frac{1}{4}\rho_{\psi} + \frac{3}{4}\rho_{\phi} \\
        &= \frac{1}{4} \Bigg( \frac{4}{5} \ket{0} \bra{0} + \frac{2}{5} \ket{0} \bra{1} + \frac{2}{5} \ket{1} \bra{0} + \frac{1}{5} \ket{1} \bra{1} \Bigg) + \frac{3}{4} \Bigg( \frac{1}{2} \ket{0} \bra{0} + \frac{1}{2} \ket{0} \bra{1} + \frac{1}{2} \ket{1} \bra{0} + \frac{1}{2} \ket{1} \bra{1} \Bigg) \\
        &= 
        \Bigg( \frac{1}{5} + \frac{3}{8} \Bigg) \ket{0} \bra{0} + \Bigg( \frac{1}{10} + \frac{3}{8}\Bigg) \ket{0} \bra{1} + \Bigg( \frac{1}{10} + \frac{3}{8} \Bigg) \ket{1} \bra{0} + \Bigg( \frac{1}{20} + \frac{3}{8} \Bigg)\ket{1} \bra{1},
\end{align*}
and it is easy to see that $Tr(\rho) = 1$.

Upon measurement, $\ket{\psi}$ is found in state $\ket{0}$ with probability $4 / 5$ and in state $\ket{1}$ with probability $1/5$. Upon measurement, $\ket{\phi}$  is found in state $\ket{0}$ with probability $1 / 2$ and in state $\ket{1}$ with probability $1/2$. 

The probability of measuring $\ket{0}$ within the ensemble is
\begin{align*}
p(0) &= \braket{0|\rho|0} \\
       &= \bra{0} \Bigg( \Bigg( \frac{1}{5} + \frac{3}{8} \Bigg) \ket{0} \bra{0} + \Bigg( \frac{1}{10} + \frac{3}{8}\Bigg) \ket{0} \bra{1} + \Bigg( \frac{1}{10} + \frac{3}{8} \Bigg) \ket{1} \bra{0} + \Bigg( \frac{1}{20} + \frac{3}{8} \Bigg)\ket{1} \bra{1} \Bigg) \ket{0} \\
       &= \Bigg( \frac{1}{5} + \frac{3}{8} \Bigg) \\
       &= \frac{23}{40},
\end{align*}
and the probability of measuring $\ket{+}$ within the ensemble is
\begin{align*}
p(1) &= \braket{1|\rho|1} \\
       &= \bra{1} \Bigg( \Bigg( \frac{1}{5} + \frac{3}{8} \Bigg) \ket{0} \bra{0} + \Bigg( \frac{1}{10} + \frac{3}{8}\Bigg) \ket{0} \bra{1} + \Bigg( \frac{1}{10} + \frac{3}{8} \Bigg) \ket{1} \bra{0} + \Bigg( \frac{1}{20} + \frac{3}{8} \Bigg)\ket{1} \bra{1} \Bigg) \ket{1} \\
       &= \Bigg( \frac{1}{20} + \frac{3}{8} \Bigg) \\
       &= \frac{68}{160} \\
       &= \frac{34}{80} \\
       &= \frac{17}{40}.
\end{align*}
\end{document}