\documentclass[10pt]{article}

\usepackage[english]{babel}
\usepackage[utf8x]{inputenc}
\usepackage{amsmath}
\usepackage{amssymb}
\usepackage{amsfonts}
\usepackage{graphicx}
\usepackage[ruled,linesnumbered,noend]{algorithm2e}
\usepackage{empheq}
\usepackage{float}
\usepackage{enumitem}
\usepackage{tikz}
\usepackage[colorlinks=true,urlcolor=blue]{hyperref}
\usepackage{braket}
\usepackage{gensymb}

\title{Kvanttilaskenta, kevät 2015 -- Viikko 2}
\author{Rodion ``rodde'' Efremov}

\begin{document}
 \maketitle

\section*{edx Problem 1}
Let 
\[
U =
\begin{pmatrix}
-2i & 5i \\
5 & 1 - i 
\end{pmatrix}.
\]
Now
\[
U^{\dag} = 
\begin{pmatrix}
2i & 5 \\
-5i & 1 + i
\end{pmatrix}.
\]

\section*{edx Problem 2}
All unitary matrices are self-inverse.

\section*{edx Problem 3}
We are given a $U$ that maps $\ket{0}$ to $\frac{1}{\sqrt{2}} \ket{0} + \frac{1+i}{2} \ket{1}$ and $\ket{1}$ to $\frac{1 - i}{2} \ket{0} - \frac{1}{\sqrt{2}} \ket{1}$. So we have that
\begin{align*}
U\ket{0} &= 
\begin{pmatrix}
a & b \\
c & d
\end{pmatrix}
\begin{pmatrix}
1 \\
0
\end{pmatrix} \\
&= 
\begin{pmatrix}
a \\
c
\end{pmatrix} \\
&= 
\begin{pmatrix}
\frac{1}{\sqrt{2}} \\
\frac{1 + i}{2}
\end{pmatrix}.
\end{align*}
Also, we have that
\begin{align*}
U\ket{1} &= 
\begin{pmatrix}
a & b \\
c & d
\end{pmatrix}
\begin{pmatrix}
0 \\
1
\end{pmatrix} \\
&= 
\begin{pmatrix}
b \\
d
\end{pmatrix} \\
&= 
\begin{pmatrix}
1-i \\
-\frac{1}{\sqrt{2}}
\end{pmatrix}.
\end{align*}
It follows immediately that
\[
U = 
\begin{pmatrix}
\frac{1}{\sqrt{2}} & 1 - i \\
\frac{1 + i}{2} & -\frac{1}{\sqrt{2}}
\end{pmatrix}.
\]

\section*{edx Problem 4}
Suppose we have a one-qubit unitary $U$ that maps $\ket{0}$ to $\frac{-3}{5} \ket{0} + \frac{4i}{5} \ket{1}$ and $\ket{+}$ to $\frac{-3-4i}{5\sqrt{2}} \ket{0} + \frac{3+4i}{5\sqrt{2}} \ket{1}$.

We have that
\begin{align*}
U\ket{0} &= 
\begin{pmatrix}
a & b \\
c & d
\end{pmatrix}
\begin{pmatrix}
1 \\
0
\end{pmatrix} \\
&= 
\begin{pmatrix}
a \\
c
\end{pmatrix} \\
&= 
\begin{pmatrix}
\frac{-3}{5} \\
\frac{4i}{5}
\end{pmatrix}.
\end{align*}
As $\ket{+} = (\frac{1}{\sqrt{2}} \quad \frac{1}{\sqrt{2}})^T$, we have that
\begin{align*}
U\ket{+} &= 
\begin{pmatrix}
a & b \\
c & d
\end{pmatrix}
\begin{pmatrix}
\frac{1}{\sqrt{2}} \\
\frac{1}{\sqrt{2}}
\end{pmatrix} \\
&= 
\begin{pmatrix}
\frac{1}{\sqrt{2}} (a + b) \\
\frac{1}{\sqrt{2}} (c + d) \\
\end{pmatrix} \\
&= 
\begin{pmatrix}
\frac{1}{\sqrt{2}} (-\frac{3}{5} + b) \\
\frac{1}{\sqrt{2}} (\frac{4i}{5} + d)
\end{pmatrix} \\
&=
\begin{pmatrix}
\frac{-3-4i}{5\sqrt{2}} \\
\frac{3+4i}{5\sqrt{2}}
\end{pmatrix}.
\end{align*}
Now for $b$ we have an equality
\[
-\frac{3}{5} + b = \frac{-3-4i}{5},
\]
which has solution $b = -\frac{4i}{5}$.
For $d$ we have an equality
\[
\frac{4i}{5} + d = \frac{3+4i}{5},
\]
which has solution $d = \frac{3}{5}$.

So the matrix in question is
\[
U = 
\begin{pmatrix}
-\frac{3}{5} & -\frac{4i}{5} \\
\frac{4i}{5} & \frac{3}{5}
\end{pmatrix}.
\]

\section*{edx Problem 4 once again - Thanks, Tomi!}
$\frac{1}{\sqrt{2}} (\ket{00} + \ket{11})$.

\section*{edx Problem 5}
$\frac{1}{\sqrt{2}} (\ket{00} - \ket{11})$.

\section*{edx Problem 6}
True.

\section*{edx Problem 5}
What is $ZX$ applied to $\ket{0}$?
We are given Pauli operators
\[
Z = \begin{pmatrix}
1 & 0 \\
0 & -1 
\end{pmatrix}, \quad
X = \begin{pmatrix}
0 & 1 \\
1 & 0
\end{pmatrix},
\]
so 
\begin{align*}
ZX\ket{0} &= \begin{pmatrix}
1 & 0 \\
0 & -1
\end{pmatrix}
\begin{pmatrix}
0 & 1 \\
1 & 0
\end{pmatrix}
\begin{pmatrix}
1 \\
0
\end{pmatrix} \\
&= \begin{pmatrix}
0 & 1 \\
-1 & 0
\end{pmatrix}
\begin{pmatrix}
1 \\
0
\end{pmatrix} \\
&= \begin{pmatrix}
0 \\
-1
\end{pmatrix} \\
&= -\ket{1}.
\end{align*}

\section*{edx Problem 6}
What is $ZX$ applied to $H\ket{0}$?

From the exerices above we have that
\[
ZX = \begin{pmatrix}
0 & 1 \\
-1 & 0
\end{pmatrix}.
\]
Also $H\ket{0} = \frac{1}{\sqrt{2}} \ket{0} + \frac{1}{\sqrt{2}} \ket{1}$, so
\begin{align*}
ZXH\ket{0} &= \begin{pmatrix}
0 & 1 \\
-1 & 0
\end{pmatrix}
\begin{pmatrix}
\frac{1}{\sqrt{2}} \\
\frac{1}{\sqrt{2}} 
\end{pmatrix} \\
&= \begin{pmatrix}
\frac{1}{\sqrt{2}} \\
-\frac{1}{\sqrt{2}}
\end{pmatrix} \\
&= \ket{-}.
\end{align*}

\section*{edx Problem 7}
We are given a qubit $\ket{\psi} = \alpha \ket{0} + \beta \ket{1}$ and we know that $|\alpha|^2 = \frac{2}{9}$ so $|\beta|^2 = \frac{7}{9}$. Also we know that
\[
H = \frac{1}{\sqrt{2}} \begin{pmatrix}
1 & 1 \\
1 & -1
\end{pmatrix},
\]
so 
\begin{align*}
H\ket{\psi} &= \frac{1}{\sqrt{2}} \begin{pmatrix}
1 & 1 \\
1 & -1
\end{pmatrix} \begin{pmatrix}
\alpha \\
\beta
\end{pmatrix} \\
&= \frac{1}{\sqrt{2}} \begin{pmatrix}
\alpha + \beta \\
\alpha - \beta
\end{pmatrix}
\end{align*}
Also we know that $\ket{+} = \frac{1}{\sqrt{2}} (\ket{0} + \ket{1})$. Now the probability of measuring $+$ is 
\begin{align*}
\cos^2 \theta &= |\braket{H\psi | +}|^2 \\
					   &= \Bigg|\frac{1}{\sqrt{2}}(\alpha + \beta)(\frac{1}{\sqrt{2}}) + \frac{1}{\sqrt{2}}(\alpha - \beta)(\frac{1}{\sqrt{2}}) \Bigg|^2 \\
					   &= \Bigg| \frac{1}{2} (\alpha + \beta) + \frac{1}{2} (\alpha - \beta) \Bigg|^2 \\
					   &= | \alpha |^2 \\
					   &= \frac{2}{9}.
\end{align*}

\section*{edx Problem 8}
All pairs commute except CNOT and X applied to the target qubit.

\section*{edx Problem 9}
Apply the 3rd (last) circuit from above.

\section*{edx Problem 10}
(a) The resulting state ain't entangled.
(b) First qubit: $b\ket{0} + a\ket{1}$, second cubit: $\ket{0}$.

\section*{edx Problem 11}
No circuit exists by no cloning theorem.

\section*{edx Problem 12}
I think in general case the correct alternative is $[0, \frac{\pi}{2}]$, yet unitary matrices preserve angles, so for any unitary $U$ $U\ket{\psi}$ $U\ket{\psi'}$ have the same angle as $\ket{\psi}$ and $\ket{\psi'}$.

\section*{edx Problem 13}
When Alice's outcome was 0, apply $I$. When Alice's outcome was 1, apply $Z$.

\section*{QCE 5.1}
Consider the following state vector:
\[
\ket{\psi} = \sqrt{\frac{5}{6}} \ket{0} + \frac{1}{\sqrt{6}} \ket{1}.
\]
(A) Is the state normalized?
As
\[
\Bigg( \sqrt{\frac{5}{6}} \Bigg)^2 + \Bigg( \frac{1}{\sqrt{6}} \Bigg)^2 = \frac{5}{6} + \frac{1}{6} = 1, 
\]
the state vector is normalized.
(B) After applying the Z-gate, we obtain
\begin{align*}
Z\ket{\psi} &= 
\begin{pmatrix}
1 & 0 \\
0 & -1
\end{pmatrix}
\begin{pmatrix}
\sqrt{\frac{5}{6}} \\
\frac{1}{\sqrt{6}}
\end{pmatrix} \\
&= 
\end{align*}
\end{document}